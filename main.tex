%Description: In this latex document now biblatex will be used for bibliography. Therefore biber must be installed: $sudo apt-get install texlive-bibtex-extra biber. Additionally if change something on the document regarding to cites you have execute the custom command " Build with Biblatex" that contains the command $pdflatex %.tex | biber % | pdflatex %.tex | evince %.pdf. Otherwise a Quick Build with PdfLatex + View PDF would be sufficient.

\documentclass[11pt,a4paper]{report} %draft zeichnet nur rahmen für bilder -> performance

% Margin
\usepackage{setspace}
\usepackage{anysize}
\marginsize{3cm}{3cm}{2cm}{2cm}

\usepackage[utf8]{inputenc}
\usepackage[german]{babel}
\usepackage[T1]{fontenc}
\usepackage{amsmath}
\usepackage{amsfonts}
\usepackage{amssymb}
\usepackage{graphicx} %for logos and images
\usepackage{caption} %for logos and tables and figures
\usepackage{csvsimple} % for creating tables from csv file
\usepackage{subcaption} %for logos
\usepackage{listings} % used for source code listings
\usepackage[svgnames]{xcolor} % color is used for background color in listings
\usepackage[colorlinks=true, linkcolor=Black,citecolor=Black,urlcolor=MidnightBlue]{hyperref} % for urls and hyperrefs -> use linkcolor to set a proper value for all links like DarkBlue or DarkGreen
\lstdefinestyle{customc}{
  belowcaptionskip=1\baselineskip,
  breaklines=true,
  frame=single,
  numbers=left,   
  firstnumber=1,
  xleftmargin=1.8em,
  language=C,
  showstringspaces=false,
  basicstyle=\footnotesize\ttfamily,
  keywordstyle=\bfseries\color{green!40!black},
  commentstyle=\itshape\color{purple!40!black},
  identifierstyle=\color{blue},
  stringstyle=\color{orange},
  tabsize=2
}
\lstdefinestyle{customshell}{
  belowcaptionskip=1\baselineskip,
  breaklines=true,
  frame=single,
  numbers=left,   
  firstnumber=1,
  xleftmargin=1.8em,
  language=C,
  showstringspaces=false,
  basicstyle=\footnotesize\ttfamily,
  keywordstyle=\bfseries\color{green!40!black},
  commentstyle=\itshape\color{purple!40!black},
  identifierstyle=\color{blue},
  stringstyle=\color{orange},
  tabsize=2
}
\lstdefinestyle{custombinarycode}{
  frame=single,
  numbers=left,   
  firstnumber=1,
  numberfirstline=false,
  xleftmargin=2.5em,
  framexleftmargin=2.5em
}

%define new chapter format!
\usepackage{titlesec}
\titleformat{\chapter}[block]{\Huge\bfseries}{\thechapter}{15pt}{\Huge\bfseries}
\titleformat{\section}[block]{\normalfont\large\bfseries}{\thesection}{0.7em}{}
\titleformat{\subsection}[block]{\normalfont\normalsize\bfseries}{\thesubsection}{0.4em}{}
%
\usepackage{csquotes} % Recommended for biblatex in combination with babel
\usepackage[backend=biber,style=numeric,sorting=ynt]{biblatex}
\addbibresource{references.bib}

\author{Johannes Busam}
\title{Masterthesis}
\date{01.09.2018}

\begin{document}
%\maketitle



\begin{titlepage}

\newcommand{\HRule}{\rule{\linewidth}{0.5mm}} % Defines a new command for the horizontal lines, change thickness here

\center % Center everything on the page
 
%----------------------------------------------------------------------------------------
%	HEADING SECTIONS
%----------------------------------------------------------------------------------------

\textsc{\LARGE Hochschule ...}\\[0.6cm] % Name of your university/college
\textsc{\Large Studiengang ...}\\ % Major heading such as course name

%----------------------------------------------------------------------------------------
%	TITLE SECTION
%----------------------------------------------------------------------------------------

\HRule \\[0.5cm]
{ \large \bfseries Masterthesis}\\[0.75cm] % Smaller one
{ \huge \bfseries Aufbau einer Plattform zur\\ forensischen Analyse basierend auf dem\\ Apache Hadoop\textsuperscript{\textregistered} Framework}\\[0.5cm] % Title
\HRule \\[0.5cm]
 
%----------------------------------------------------------------------------------------
%	AUTHOR SECTION
%----------------------------------------------------------------------------------------

\Large Zur Erlangung des akademischen Grades\\
\Large Master of Science\\[0.5cm] 

\Large vorgelegt im Sommersemester 2018\\[0.5cm] 

\Large von\\ Johannes Busam 

\begin{flushleft}
Erstbetreuung: ...\\[0.3cm] 

\noindent
Zweitbetreuung: ...\\

\end{flushleft}

\vfill % Fill the rest of the page with whitespace

\end{titlepage}

\tableofcontents

%----------------------------------------------------------------------------------------
%	Einleitung
%----------------------------------------------------------------------------------------
\chapter{Einleitung}
\label{ch:einleitung}
\section{Allgemeines}

\section{Problemstellung}
Die forensische Analyse von digitalen Beweismitteln ist in der heutigen Zeit ein wichtiger Aspekt, um in der Strafverfolgung rechtswidriges Verhalten aufzudecken oder nachzuweisen. In vielen Fällen werden informationstechnische Systeme am Tatort gefunden oder zur Tatbegehung genutzt. Einschlägig sind hierbei Angriffe auf kritische Infrastrukturen durch Computersabotage oder das Ausspähen von Daten. Aber auch Urheberrechtsverletzungen durch die Weitergabe von geschützten Medien oder Verstöße gegen das Wettbewerbsrecht werden mit Informationstechnik begangen.
Je nach Dauer und Umfang der Strafhandlung werden gerade auch im Bereich der Wirtschaftskriminalität dutzende Beweismittel von informationstechnischen Systemen erhoben. Beispielsweise werden beteiligte Computer und Mobiltelefone sichergestellt. Oder es werden logische Sicherungen von Netzwerkspeichern durchgeführt.\\

\noindent
Bei der Analyse dieser Beweismittel möchte ein forensischer Ermittler möglichst schnell einen Überblick über die sichergestellten Daten erhalten. Darauf aufbauend kann er entscheiden, welche Spuren in den Daten zum Nachweis konkreter Tathandlungen dienen und welche potentielle Beweismittel nicht weiter analysiert werden müssen.\\

\noindent
Der kritischste Aspekt hierbei ist, in kürzester Zeit die richtigen Informationen aus allen Daten zu extrahieren. Denn gerade in der Strafverfolgung ist eine schnelle und zielgerichtete Aufarbeitung der Ermittlungsfälle erforderlich. Darüber hinaus werden während der Analyse oftmals weitere Indizien gefunden, welche wiederum zur Sicherung neuer Beweismittel führen können. Je mehr Zeit jedoch für die Analyse benötigt wird, desto höher ist die Gefahr, dass noch nicht sichergestellte Daten endgültig gelöscht werden. Beispielsweise werden Telekommunikationsverbindungsdaten nicht über längere Zeiträume gespeichert.\\

\noindent
Zur Analyse stehen dem Forensiker etliche kommerzielle und Open Source Programme zur Auswahl. Allerdings sind im forensischen Open Source Bereich viele Programme durch die Ressourcen des Analyserechners beschränkt. Sie bieten keine Möglichkeiten rechenintensive Aufgaben performant auf mehreren Computern zu skalieren.\\

\noindent
Aus fachlicher Sicht wäre eine Plattform sinnvoll, die anfallende Analyseaufgaben automatisiert auf allen Daten durchführt. Das System sollte die Ergebnisse unter Berücksichtigung verfügbarer Ressourcen schnellstmöglich ermitteln und dem forensischen Ermittler in einer aufbereiteten Form darstellen. Auf Basis dieser Ergebnisse könnte sich der Forensiker möglichst frühzeitig einen Überblick aller Beweismittel verschaffen, um dann bestimmte Daten und Informationen auch in anderen spezialisierten Analysetools weiterzuverarbeiten. 

\section{Zielsetzung}
Zur Lösung der Problemstellung soll in dieser Masterthesis eine Plattform zur forensischen Analyse entwickelt werden. Diese Plattform soll durch eine automatisierte Analyse und Aufbereitung forensisch relevanter Informationen dem Nutzer helfen, sich einen Überblick zu verschaffen. Der Forensiker soll dadurch effizient und zielgerichtet Datenanalysen durchführen können. Als Basis dieser Plattform soll das Apache Hadoop\textsuperscript{\textregistered} Framework genutzt werden. Hierbei sollen Vor- und Nachteile dieser Art der Datenverarbeitung im forensischen Kontext herausgearbeitet werden.\\ 

\noindent
Apache Hadoop ist ein etabliertes Open Source Framework zur verteilten Speicherung und Verarbeitung von Daten. Durch die Verwendung paralleler Algorithmen eignet sich ein Hadoop-Cluster für große Datenmengen im Terabyte-Bereich. Ein zugrunde liegendes Paradigma ist hierbei, dass die Programmausführung dort stattfindet wo auch die Daten liegen, um kostspielige Datentransporte weitgehend zu vermeiden. Aufgrund dieser Beschaffenheit könnte diese Art der Datenverarbeitung auch Geschwindigkeitsvorteile bei forensischen Analysen bieten. Das Framework selbst besteht aus mehreren Komponenten, welche spezifische Aufgaben der Datenverarbeitung übernehmen. \\

\noindent
Die Basis bildet das verteilte Dateisystem \textit{HDFS}\footnote{HDFS ist die Abkürzung für \textit{Hadoop Distributed File System}.}, welches die Daten redundant auf allen Knoten des Computer-Clusters speichert. Der Zugriff auf die Daten kann über unterschiedliche Komponenten erfolgen.\\
So findet beispielsweise meist Apache Spark\texttrademark\enspace bei der Prozessierung und Analyse der Daten Anwendung. Auch in der Masterthesis soll Apache Spark für anfallende Analysezwecke verwendet werden. Hierbei übernimmt der Ressourcenmanager \textit{YARN}\footnote{YARN ist die Abkürzung für \textit{Yet Another Resource Negotiator}.} die Bereitstellung und Verteilung von verfügbarer Rechenleistung.\\

\noindent
Ein Knackpunkt der Masterthesis ist die Aufbereitung der Daten für die Analyse im Hadoop-Cluster. Beispielsweise könnten die Dateien von sichergestellten Datenträgern direkt in das HDFS kopiert werden. Hierbei muss auf die Unversehrtheit der Dateiinhalte und Metadaten beim Kopieren geachtet werden. \\

\noindent
Im Rahmen dieser Thesis soll die reine Datenanalyse vorerst auf grundlegende Operationen basieren. Unter anderem sollen beispielsweise folgende Informationen zu einzelnen Dateien ermittelt werden: Name, vollständiger Pfad, Hashsumme, Dateityp, Größe, Zeitpunkt der letzten Änderung und Erstellung.\\
Es soll auch eine Volltextsuche auf den Daten möglich sein. Darauf aufbauend soll der Nutzer beispielsweise gleiche Dateien und Verbindungen zwischen den einzelnen Beweismitteln erkennen können.\\
Optional könnte die Analyseplattform gezielt nach IP-Adressen, URLs, E-Mail-Adressen oder Positionsdaten\footnote{Beispielsweise könnten Geopositionen oder Ortsnamen aus Dateien extrahiert werden. Diese Daten könnten dann mit ihrem geografischen Bezug auf einer Karte dargestellt werden.} suchen. Diese Operationen sollen erst implementiert werden, wenn im Rahmen der Thesis noch weitere Bearbeitungszeit vorhanden ist. Sie sind vorerst nicht Gegenstand der Thesis.\\

\noindent
Die Resultate durchgeführter Datenanalysen sollen dem Nutzer bereitgestellt werden. Hierzu soll eine prototypische Implementierung entwickelt werden, deren grafische Oberfläche die fachlichen Aspekte der forensischen Analyse widerspiegelt. Der Forensiker soll Analyseaufgaben konfigurieren und starten können. Nach der Prozessierung soll er die Resultate der Analysen direkt einsehen können.\\
Allerdings ist es nicht das Ziel dieser Thesis, detaillierte Konfigurationsmöglichkeiten und unterschiedlichste Visualisierungen zu implementieren. Dies würde den Rahmen der Arbeit übersteigen. Der Fokus dieser Fachanwendung liegt bei der schlichten Anzeige der Analyseergebnisse. In diesem Kontext soll auch geprüft werden, ob existierende Programme zur Datenvisualisierung im Hadoop-Umfeld wiederverwendet werden können.\\

\noindent
Nachfolgende Abbildung soll den den groben Aufbau dieser Plattform skizzieren. Der Forensiker importiert die forensischen Rohdaten in das Hadoop-Cluster. Darauf hat er die Möglichkeit diverse Analysen auf den Daten durchzuführen. Zuletzt kann er die Ergebnisse über eine entsprechende Oberfläche einsehen und hat die Möglichkeit die Daten innerhalb oder außerhalb des Hadoop-Clusters weiterzuverarbeiten.\\

\begin{figure}[ht]
  \centering
  \includegraphics[width=0.82\textwidth]{./resource/Hadoop-Struktur.pdf}
  \caption{Datenverarbeitung im Hadoop-Umfeld}
\end{figure}

\noindent
Das Ziel dieser Masterthesis ist es, dem forensischen Ermittler schnellstmöglich einen Überblick zu den einzelnen Beweismitteln und deren Zusammenhänge im Kontext einer Fallanalyse zu liefern. 

\section{Weitere Aspekte der Thesis}
Bei einer realen forensischen Analyse gibt es weitere Anforderungen, die das Analysesystem erfüllen sollte. Im Rahmen der Masterthesis soll geprüft werden, ob diese Anforderungen durch die Nutzung des Apache Hadoop Frameworks abgedeckt werden und welche technischen oder organisatorischen Regelungen getroffen werden müssen.\\
Das System muss gegen fremden Zugriff gesichert sein. Es muss zu jeder Zeit ersichtlich sein, welche Personen zu welchem Zweck auf das System zugreifen.\\

\noindent
Da in vielen Fällen hochsensible personenbezogene Daten und Geschäftsgeheimnisse verarbeitet werden, müssen auch entsprechende Regelungen getroffen werden, wie nach der Analyse alle Daten restlos aus dem System gelöscht werden können.\\

\noindent
Ein weiterer Aspekt in der Analyse ist die lückenlose Erstellung einer Beweismittelkette (Chain of Custody). Für jedes forensische Analyseergebnis müssen die Herkunft und die Verarbeitungsschritte transparent nachvollziehbar sein. \\

\noindent
Auch die Korrektheit der Analyseergebnisse muss verifiziert werden. Hierfür sollen im Rahmen der Masterthesis entsprechende Testdaten erstellt oder beschafft werden, welche die Funktionsfähigkeit der Plattform prüfen. \\

\noindent
Aus organisatorischer Sicht soll die Analyseplattform als Open Source Projekt bereitgestellt werden. Hierzu soll der Source-Code, die Konfiguration des Systems und die Dokumentation in einem öffentlich zugänglichen Repository verfügbar sein.\footnote{Sicherheitskritische Informationen, wie beispielsweise Zugangsdaten, müssen unkenntlich gemacht werden.}


\section{Aufbau}
In Kapitel \ref{ch:einleitung} wird die grundlegende Problemstellung bei der forensischen Analyse beschrieben, welche diese Masterthesis lösen soll. Darauf folgt die Zielsetzung der Thesis, welche als möglicher Lösungsvorschlag zur beschriebenen Problemstellung gilt.\\
In Kapitel \ref{ch:development_approach} folgt das allgemeine Entwicklungsvorgehen. Darin ist auch der aktuelle Projektplan enthalten, welche die Arbeitspakete definiert.
Zusätzlich wird der Umgang mit Quellcode und Konfigurationsdateien als Open-Source Projekt beschrieben.\\

\noindent
In Kapitel \ref{ch:theory_hadoop} erfolgt eine Darstellung der Apache Hadoop Plattform inklusive theoretischen Grundlagen zur Arbeitsweise des Frameworks. Des Weiteren werden darauf aufbauend Projekte Apache Spark, Apache Hive und Apache HBase und deren Einsatzbereiche erläutert.\\

\noindent
In Kapitel \ref{ch:praxis_analysis} werden die angewendeten Tools für eine herkömmliche fachliche Analyse eines Beweismittels beschrieben. Parallel hierzu wird bei jedem herkömmlichen Programm geprüft, wie das gleiche Ergebnis mit der Analyseplattform erzielt werden kann. Diese Kapitel soll sozusagen die Überleitung von der herkömmlichen Analyse auf einem Computer hin zu Analyse im Hadoop-Cluster beschrieben. Als Ergebnis soll anschaulich dargestellt werden, welche fachliche Analyse-Schritte mithilfe eines Hadoop-Cluster sinnvoll und performant durchgeführt werden können.\\

\noindent
Zuletzt erfolgt in Kapitel \ref{ch:zusammenfassung} eine Zusammenfassung der erarbeiteten Ergebnisse. Offene Punkte und Verbesserungen des Systems werden in Kapitel \ref{ch:ausblick} diskutiert. 


\chapter{Vorgehen}
\label{ch:development_approach}

\section{Projektplanung}
\label{sec:project_plan}
Abbildung \ref{fig:workpackages} zeigt die Aufteilung Masterthesis in einzelne Arbeitspakete. Das Ziel der Einarbeitungsphase ist ein grundlegendes Verständnis über die Datenverarbeitung im Hadoop-Framework zu erhalten. Zusätzlich soll eine Entwicklungsumgebung inklusive öffentlicher Versionsverwaltung eingerichtet werden. Darauf erfolgt der Aufbau eines eigenen Hadoop-Clusters und die Beschaffung von Testdaten.\footnote{Auch ein Zugriff auf einen bestehendes Hadoop-Cluster ist möglich.} Für die Einarbeitung und den Aufbau sind vier Wochen eingeplant (siehe Abbildung \ref{fig:ganttA}).\footnote{Die referenzierten Gantt-Diagramme wurden mit der JavaScript-Bibliothek \textit{dhtmlxGantt} erstellt. Der Quellcode ist unter der \textit{GNU GPLv2}-Lizenz lizenziert. Weiter Informationen können in Kapitel \ref{sec:licencing_issues} im Anhang nachgelesen werden.}\\

\noindent
Der zweite Teil behandelt die Rohdatenspeicherung im HDFS und eine Datenaufbereitung. Es soll geprüft werden, welche Struktur der Daten für eine optimale Speicherung und Verarbeitung im Hadoop-Framework erforderlich ist. Dieser Teil beansprucht abermals vier Wochen. Am Ende dieses Arbeitspaketes soll ein erster Zwischenbericht erstellt werden, welcher die bisherigen Ergebnisse enthält (sieh Abbildung \ref{fig:ganttA}).\\

\begin{figure}[ht]
  \centering
  \includegraphics[width=\textwidth]{./resource/Arbeitspakete.pdf}
  \caption{Arbeitspakete der Masterthesis}
  \label{fig:workpackages}
\end{figure}

\noindent
Nach der Speicherung der Rohdaten erfolgt im dritten Arbeitspaket die Datenanalyse mit Apache Spark. Hier sollen die Daten nach anwendungsbezogenen Problemstellungen analysiert werden. Das Ergebnis ist eine Sammlung von Programmen, welche mit Apache Spark auf den Daten ausgeführt werden können. Darüber hinaus soll ermittelt werden, welche Möglichkeiten zur Ausführung dieser Spark-Anwendungen bestehen.\footnote{Hier könnte beispielsweise das Projekt Apache Livy nützlich sein.} Ein weiterer Aspekt der Datenanalyse beschäftigt sich mit den Möglichkeiten, wie die Ergebnisse persistiert werden können.\footnote{Hier könnten die
Projekte Apache Hive und Apache HBase zur Speicherung von strukturierten und unstrukturierten Daten untersucht werden.} Im Anschluss soll die Performanz der Algorithmen geprüft werden. Hier bietet sich der Vergleich zu herkömmlichen Analyseprogrammen an. Denn schließlich hat diese Thesis auch das Ziel, bei großen Datenmengen schneller Ergebnisse zu liefern als die herkömmlichen Analysewerkzeuge auf einem einzelnen Analyserechner. Für dieses Arbeitspaket sind sieben Wochen eingeplant (siehe Abbildung \ref{fig:ganttB}). Darauf folgt ein zweiter Zwischenbericht.\\

\noindent
Im letzten Drittel der Masterthesis sollen die querschnittlichen Aspekte in der bestehenden Datenverarbeitung berücksichtigt werden. Hierbei geht es um das Absichern der Analyseplattform, die Dokumentation der Chain of Custody und das Löschen von nicht mehr verwendeten personenbezogenen Daten. Für dieses Arbeitspaket sind vier Wochen eingeplant (siehe Abbildung \ref{fig:ganttC}).\\

\noindent
Das letzte Arbeitspaket enthält ein prototypische Visualisierung der Analyseergebnisse. Hierbei soll geprüft werden, welche Möglichkeiten zur Darstellung der Ergebnisse existieren. Der Forensiker soll auf möglichst einfache Art und Weise die Ergebnisse ansehen können. Für diese Arbeit sind drei Wochen eingeplant (siehe Abbildung \ref{fig:ganttC}).\\

\begin{figure}[p]
  \centering
  \includegraphics[width=\textwidth,height=\textheight,keepaspectratio]{./resource/ganttA.png}
  \caption{Projektplan Teil A - Einarbeitung und Rohdatenspeicherung (siehe Kapitel \ref{sec:licencing_issues})}
  \label{fig:ganttA}
\end{figure}

\begin{figure}[p]
  \centering
  \includegraphics[width=\textwidth,height=\textheight,keepaspectratio]{./resource/ganttB.png}
  \caption{Projektplan Teil B - Datenanalyse (siehe Kapitel \ref{sec:licencing_issues})}
  \label{fig:ganttB}
\end{figure}

\begin{figure}[p]
  \centering
  \includegraphics[width=\textwidth,height=\textheight,keepaspectratio]{./resource/ganttC.png}
  \caption{Projektplan Teil C - Querschnittliche Aspekte und Visualisierung (siehe Kapitel \ref{sec:licencing_issues})}
  \label{fig:ganttC}
\end{figure}

\clearpage
\section{Entwicklungsumgebung}
\label{development_environment}
Der Aufbau eine Test- und Entwicklungsumgebung ist ein essentieller Bestandteil dieser Thesis. Einerseits sollen Anwendungsprogramme zur Datenverarbeitung schnell und lokal ausführbar sein und andererseits, soll die Testumgebung auf einem physikalisches Apache Hadoop Cluster laufen, um mögliche Infrastrukturproblem identifizieren zu können. \\

\noindent
Alle selbst erstellten Anwendungsprogramme, Konfigurationsdateien und die Dokumentation dieser Thesis sollen als Open Source - Quelle in einem öffentlichen Repository zugänglich sein. Aus fachlicher Sicht ist es gerade in der Forensik sehr wichtig, dem Nutzer die Möglichkeit zu geben, den Quellcode der Analyseprogramme einsehen zu können und in notfalls auf spezielle Bedürfnisse anzupassen. Darüber hinaus kann die Datenverarbeitung transparent nachvollzogen werden.\\

\noindent
Aus diesem Grund soll sämtlicher Source-Code unter der Apache License vertrieben werden. Es soll jedem möglich sein, die Dateien einzusehen. Darüber hinaus wird das Projekt vorerst unter GitHub gehostet.


%----------------------------------------------------------------------------------------
%	Ergebnisse
%----------------------------------------------------------------------------------------

\chapter{Grundlagen von Apache Hadoop\textsuperscript{\textregistered}}
\label{ch:theory_hadoop}

\chapter{Grundlagen von Apache Hadoop\textsuperscript{\textregistered}}
\label{ch:theory_hadoop}

\section{Apache Hadoop Framework}
\label{sec:theory_hadoop}
\noindent
Apache Hadoop\textsuperscript{\textregistered} ist ein etabliertes Java-Framework zur verteilten Speicherung und Verarbeitung von Daten. Durch die parallele Ausführung von Algorithmen eignet sich ein Hadoop-Cluster für rechenaufwendige Datenanalysen. Ein primäres Paradigma ist das Konzept der \textit{Datenlokalität}. Die auszuführenden Programme werden auf die Knoten verteilt, auf welchen auch die Daten liegen. Ressourcenintensive Datentransporte sollen weitgehend vermieden werden.\cite[S. 20 ff.]{big_data_praxis}\\ 
Das Framework ist für die Ausführung auf Standardhardware konzipiert. Es wird also keine verhältnismäßig teure Spezialhardware benötigt. Das Cluster besteht aus vielen einzelnen Knoten mit Standardhardware, welche im Verhältnis zu Spezialhardware günstiger und leicht ersetzbar ist. Der Ausfall einzelner Knoten ist die Regel und wird bei der Datenhaltung entsprechend berücksichtigt. \\

\noindent
Hadoop selbst besteht aus mehreren Komponenten, welche spezifische Aufgaben übernehmen. Abbildung \ref{fig:hadoop_framework_structure} stellt eine grobe Skizzierung der Komponentenlandschaft der Plattform dar.\footnote{In der Abbildung \ref{fig:hadoop_framework_structure} werden Logos der einzelnen Apache Projekte verwendet. Diese sind Handelsmarken der \textit{Apache Source Foundation} (siehe \url{https://www.apache.org/}). In Kapitel \ref{sec:licencing_issues} im Anhang werden die Logos und deren Herkunft nochmals aufgelistet.}\\

\begin{figure}[ht]
  \centering
  \includegraphics[width=\textwidth]{./resource/hadoop_framework_structure.pdf}
  \caption{Apache Hadoop Ökosystem (Vgl. \cite{big_data_praxis},\cite{expert_hadoop_admin}. Siehe Kapitel \ref{sec:licencing_issues})}
  \label{fig:hadoop_framework_structure}
\end{figure}

\noindent
Die Basis bildet das \textit{\gls{hdfs}}, welches die Daten redundant auf allen Knoten des Computer-Clusters speichert. Hierbei besteht das Computer-Cluster selbst aus mehreren Knoten, auf welchen vorzugsweise ein Linux-Betriebssystem, wie beispielsweise CentOS, läuft.\\
Der Ressourcenmanager \textit{\acrshort{yarn} (\acrlong{yarn})} ist für die Verteilung und Bereitstellung von verfügbarer Rechenleistung verantwortlich.\\ 
Die dritte Komponente ist das \textit{Hadoop Map-Reduce Framework}. Hadoop Map-Reduce kann zur Datenverarbeitung genutzt werden. Hierbei werden Algorithmen parallel auf den Knoten prozessiert und die Ergebnisse im Anschluss zusammengetragen. Die einzelnen Zwischenergebnisse werden alle im \gls{hdfs} abgelegt.\footnote{Sogenannte Map-Reduce Jobs bildeten in den Anfängen von Hadoop den primären Weg, Daten verteilt zu verarbeiten. Mittlerweile wurde diese Art der Datenverarbeitung in den Hintergrund verdrängt, da andere Projekte, wie beispielsweise Apache Spark, die Daten schneller verarbeiten können oder andere Ansätze zur Verarbeitung nutzen. Dies ist beispielsweise auch der Grund, weshalb Hadoop Map-Reduce in dieser Masterthesis nicht genutzt wird.} \\

\noindent
Das verteilte Dateisystem \gls{hdfs} und der Ressourcenmanager \acrshort{yarn} bilden den Kern des Hadoop-Clusters. Darauf aufbauend können andere Komponenten die Daten verarbeiten oder weitere Funktionen anbieten.
So wird beispielsweise in dieser Thesis \textit{Apache Spark\texttrademark\thinspace} bei der Prozessierung und Analyse der Daten genutzt. Der Vorteil von Apache Spark ist eine performante Datenverarbeitung, da einerseits die Daten verteilt verarbeitet werden und andererseits Zwischenergebnisse und temporäre Daten im Arbeitsspeicher der einzelnen Rechenknoten gehalten werden.\footnote{Durch das In-Memory Computing ist Apache Spark deutlich schneller als das bereits erwähnte Hadoop Map-Reduce.}\\
Des Weiteren bietet \textit{Apache Hive\texttrademark\thinspace} eine Möglichkeit Dateien im \gls{hdfs} mithilfe einer \acrshort{sql} ähnlichen Syntax (HiveQL) abzufragen. Hierbei nutzt die Komponente wiederum das Map-Reduce Framework von Hadoop. Apache Hive ist jedoch keine reine Datenbank, sondern arbeitet auf den Dateien im \gls{hdfs}.\\
\textit{Apache HBase\textsuperscript{\textregistered}} hingegen ist eine spaltenorientierte Key-Value Datenbank. Sie wurde eigens für Apache Hadoop implementiert, 
um große Datenmengen performant zu speichern.\\

\noindent
Das Hadoop-Ökosystem als Ganzes muss auch konfiguriert und überwacht werden. Um die Verfügbarkeit einzelner Instanzen zu gewährleisten und gegebenenfalls redundante Verarbeitungswege anzubieten, wird \textit{Apache ZooKeeper\texttrademark\thinspace} genutzt. Mit ZooKeeper ist es auch möglich Konfigurationen und Änderungen im Cluster zu verteilen. Zum eigentlichen konfigurieren und überwachen des Hadoop-Clusters wird \textit{Apache Ambari\texttrademark\thinspace} genutzt.\\

\noindent
Zusätzlich existieren dutzende weitere Projekte die auf dem Hadoop-Ökosystem aufbauen oder sich integrieren lassen. Mit \textit{Apache Accumulo\textsuperscript{\textregistered}} existiert eine weitere Key-Value Datenbank, welche eine Alternative zu HBase bietet. \textit{Apache Livy} kann zur Ausführung von Apache Spark Anwendungen über eine \acrshort{rest}-Schnittstelle genutzt werden.\footnote{\textit{\gls{rest}} bezeichnet ein Programmierparadigma in verteilten Systemen. Hierbei werden Ressourcen über das \gls{http} angefordert, gespeichert und verarbeitet.} \textit{Apache NiFi} hingegen ermöglicht das Aufbereiten von Daten und organisiert Datenimporte.\\
Im Rahmen dieser Thesis wird auch das bekannte Open-Source Projekt \textit{Apache Solr\texttrademark\thinspace} verwendet, um innerhalb des Hadoop-Ökosystems eine Datenindexierung für eine Volltextsuche durchzuführen.\\


\noindent
Prinzipiell sind viele Komponenten unabhängig voneinander. So kann ein \gls{hdfs} ausschließlich zur Datenhaltung aufgebaut werden, ohne eine Komponente zur Datenverarbeitung verwenden zu müssen. Umgekehrt lassen sich Komponenten zur Datenverarbeitung, wie Apache Spark, auch ohne das \gls{hdfs} und YARN nutzen und könnten damit auch in andere Umgebungen integriert werden. Die einzelnen Komponenten entfalten jedoch gerade durch die Kombination miteinander ihre Potential zur performanten Datenanalyse.\\

\noindent
Es gibt einige Unternehmen, die sich darauf spezialisiert haben das Hadoop-Ökosystem inklusive weiterer Komponenten zu einzelnen Analyseplattformen zusammenzufassen. Sie bieten dafür kostenpflichtigen Support an, wobei diese Plattformen auch kostenfrei betrieben werden können. So wird im Praxisteil der Masterthesis beispielsweise die \textit{\gls{hdp}} des Unternehmens \textit{Hortonworks} genutzt.


\section{Apache Hadoop HDFS}
\label{sec:theory_hdfs}
Das \acrfull{hdfs} ist ein verteiltes Dateisystem, welches die Grundlage zu Speicherung von Daten im Hadoop-Ökosystem bietet. Nachfolgende Zwecke soll es erfüllen.\\
Es soll ausfallsicher sein. In der Standardkonfiguration wird jede Datei im \gls{hdfs} dreifach auf unterschiedlichen physikalischen Knoten gespeichert. Damit kann selbst bei einem Ausfall von zwei Knoten immer noch auf die Datei zugegriffen werden. Darüber hinaus verteilt das \gls{hdfs} die Dateien automatisch und regeneriert sich selbst nach einem Knotenausfall. In großen Computer-Clustern mit mehreren hunderten Knoten ist ein Ausfall eines Knoten kein Sonderfall sondern die Regel. Daher muss es sich selbst heilen können, um auch ohne manuelle Administration weiter verfügbar zu sein.\\
Das \gls{hdfs} (und auch Hadoop im Allgemeinen) soll horizontal skalierbar sein. Wird mehr Speicher benötigt, sollen einfach noch Knoten hinzugefügt werden können.\\
Das \gls{hdfs} ist auf hohen Datendurchsatz und die Speicherung großer Datenmengen ausgelegt.
So können einzelne Dateien mehrere Gigabyte bis hin zu Terabyte groß sein und es können mehrere Millionen Dateien im \gls{hdfs} gespeichert werden. Die Optimierung auf einen möglichst hohen Datendurchsatz geht mit einer schlechteren Reaktionszeit im Vergleich zu herkömmlichen Dateisystemen einher.\\
Das Prinzip \textit{Write-once-Read-many} wird im \gls{hdfs} implementiert. Wenn Daten einmal geschrieben wurden, dann werden sie normalerweise nicht mehr geändert. Dies ermöglicht ein einfacheres Koherenzmodell. Dies fördert den Lesedurchsatz indem die Unterstützung der Modifikation von Daten stark eingeschränkt wird. Ein wahlfreies Schreiben in eine existierende Datei wird beispielsweise nicht unterstützt. Änderungen an Daten, welche von Algorithmen vorgenommen werden, resultieren in neuen Datensätzen.\\
Darüber hinaus gilt das Prinzip der Datenlokalität. Algorithmen werden dort ausgeführt, wo die Daten liegen, um das Transportieren von Daten über das Netzwerk zu vermeiden.\cite{hdfs_architecture}\\

\noindent
Der Aufbau eines \gls{hdfs} bildet eine Master-Slave Architektur aus einem \textit{Name Nodes} und mehreren \textit{Data Nodes}. Der Name Node ist einmalig im verteilten System vorhanden und enthält alle Metainformationen zu den Dateien. Eine Datei selbst wird in ein oder mehrere Blöcke aufgeteilt und auf mehreren Data Nodes gespeichert. Der Name Node organisiert diese Speicherung und bestimmt, wo welche Daten persistiert werden. Über den Name Node selbst fließen aber keine Rohdaten von Dateiinhalten. Auf Dateisystemebene ist das \gls{hdfs} wie gängige Dateisysteme hierarchisch organisiert. Jede Datei wird über einen absoluten Pfad eindeutig bestimmt und erhält entsprechende Metadaten, wie Dateirechte und Zeitstempel.  \\

\noindent
Abbildung \ref{fig:hdfs_cluster_architecture} verdeutlicht die Struktur im \gls{hdfs}. Angenommen es soll eine Datei im \gls{hdfs} unter \path{/home/foo.txt} gespeichert werden. Dazu kann ein HDFS-Client genutzt werden, welcher Zugang zum Hadoop-Cluster hat. Der HDFS-Client speichert zuerst die Metadaten der Datei auf dem Name Node. Der Name Node kennt die Größe der Datei und entscheidet, in viele Blöcke sie unterteilt werden soll. Er ermittelt für jeden einzelnen Block, auf welchen Data Nodes dieser Block gespeichert werden soll. Diese Blockaufteilung und die Zuordnung zu den Data Nodes werden an den HDFS-Client zurückgeschickt. Dieser übermittelt die Blöcke an einen der Data Nodes. Sobald ein Data Node einen Block empfangen hat, schickt er diesen Block auch an die Knoten, welche eine Replikation des Blocks speichern sollen. Die Data Nodes stehen in Kontakt zum Name Node und senden Informationen über ihren Zustand und die momentan gespeicherten Blöcke. Der Name Node bekommt dadurch auch mit, wann ein Data Node ausfällt.

\begin{figure}[ht]
  \centering
  \includegraphics[width=\textwidth]{./resource/hdfs_cluster_architecture.pdf}
  \caption{HDFS - Datenspeicherung im Verbund (Vgl. \cite{hdfs_architecture},\cite{expert_hadoop_admin})}
  \label{fig:hdfs_cluster_architecture}
\end{figure}

\noindent
Ein Block ist in der Standardkonfiguration 128 MB groß. Er kann bis zu 512 MB Blockgröße konfiguriert werden. Bei der Replikation wird der gleiche Block in unterschiedlichen Data Nodes angelegt.\\
Es ist nicht erlaubt den gleichen Block mehrmals im gleichen Data Node zu replizieren. Dies würde ja auch keinen Sinn ergeben, da die Replikation vor Datenverlust bei Ausfällen von einzelnen Knoten schützen soll. Wichtig hierbei ist auch, dass im Produktivsystem auf jedem physikalischen Knoten auch nur ein Data Node oder ein Name Node läuft. Denn würden beispielsweise mehrere Data Nodes auf dem gleichen physikalischen Knoten laufen, so wäre bei einem Ausfall nicht mir garantiert, dass die Dateiinhalte auch noch auf mindestens zwei anderen verfügbaren Knoten gespeichert sind. Denn der Replikationsmechanismus im \gls{hdfs} kann nicht erkennen, ob jeder Knoten physikalisch unabhängig arbeitet. Allerdings hat Hadoop eine sogenannte \textit{Rack-Awareness}. So ist es möglich zu bestimmen, welche physikalischen Knoten in einem gemeinsamen Rack gruppiert sind. Abhängig davon, versucht das \gls{hdfs} die Daten teilweise im selben Rack redundant zu speichern aber auch einige Replikationen außerhalb des Racks anzulegen. So kann auch der Ausfall eines Racks im Notfall kompensiert werden.\\
In einzelnen Testumgebungen ist es aber durchaus möglich einen Name Node und einen Data Node oder mehrere Data Nodes gemeinsam auf einem physikalischen Knoten zu installieren. Allerdings greifen die Mechanismen für eine Toleranz gegenüber Hardwareausfällen dann nicht mehr.\\

\noindent
Wie in Abbildung \ref{fig:hdfs_cluster_architecture} ersichtlich, ist der Name Node die Schlüsselstelle im HDFS-Cluster. Zusätzlich existiert ein sogenannter \textit{Secondary Name Node}, der den (First) Name Node beim Speichern von Daten in gewisser Hinsicht unterstützt. Der Name Node hält die Metainformationen des \gls{hdfs} im Arbeitsspeicher. Zusätzlich existieren zwei Dateien, das sogenannte \textit{FsImage} und ein \textit{EditLog}, welche persistent auf der Festplatte gespeichert sind. Das FsImage selbst beschreibt einen Zustand der Dateisystemmetainformationen zu einem gewissen Zeitpunkt (Checkpoint). Im EditLog befinden sich alle Änderungen seit dem letzten Checkpoint bis zum aktuellen Zeitpunkt. Der Secondary Name Node erstellt aus dem FsImage und dem EditLog regelmäßig neue Checkpoints, welche wiederum als neu FsImages gespeichert werden.
Dieser Mechanismus wurde implementiert, um bei einem Neustart des Name Nodes die Startzeit zu optimieren. Dadurch kann die Verfügbarkeit des gesamten \gls{hdfs}  verbessert werden. Der \textit{Secondary Name Node} unterstützt also den (First) Name Node, er kann ihn aber nicht ersetzen. Bei einem Ausfall wäre das \gls{hdfs} nicht mehr einsatzbereit. Um dieses Problem zu umgehen, kann ein sogenannter \textit{Standby Name Node} konfiguriert werden. Dieser kann einspringen, sobald der erste Name Node ausgefallen ist. Allerdings muss er extra konfiguriert werden.\cite[S. 40 ff.]{expert_hadoop_admin}\\

\noindent
Das \gls{hdfs} selbst kann über mehrere Wege genutzt werden. Es gibt eine Kommandozeilenschnittstelle, die sogenannte \textit{FS Shell}. Es ist auch möglich über eine Java oder C++-Schnittstelle Datenzugriff zu erhalten. Das Dateisystem kann auch über eine \gls{rest}-Schnittstelle via \gls{http}(S) genutzt werden. Auch das Mounten als \textit{\gls{nfs}} ist möglich.

\section{Apache Hadoop YARN}
\label{sec:theory_yarn}
\acrshort{yarn} (\acrlong{yarn}) ist eine Ressourcenverwaltung, welche die verfügbaren Ressourcen innerhalb des Hadoop Clusters organisiert und die Ausführungsreihenfolge von Jobs plant und überwacht. Es gibt einen separaten \textit{Resource Manager}, welcher nur die Ressourcen verwaltet. Auf jedem Knoten, der auch Datenverarbeitungen durchführt, ist ein \textit{Node Manager} installiert. Zuletzt gibt es noch einen \textit{Application Manager} für jeden einzelnen Job, der ausgeführt werden soll. Der Application Manager kontrolliert die Ausführung des Jobs.\\
Abbildung \ref{fig:yarn_cluster_architecture} zeigt die Komponenten von YARN im Cluster.\\

\begin{figure}[ht]
  \centering
  \includegraphics[width=\textwidth]{./resource/yarn_cluster_architecture.pdf}
  \caption{Ressourcenverteilung mit YARN (Vgl. \cite{yarn_architecture},\cite{expert_hadoop_admin})}
  \label{fig:yarn_cluster_architecture}
\end{figure}

\noindent
Ein Job oder eine Anwendung besteht aus mehreren Tasks. Diese Tasks können parallel in mehreren sogenannten Container ausgeführt werden. Ein Container ist eine abstrakte parallele Verarbeitungseinheit, welche bestimmte \acrshort{cpu}- und Speicher-Ressourcen enthält. Es können mehrere dieser Container auf einem Knoten innerhalb des Clusters ausgeführt werden. Beispielsweise werden bei einem Knoten mit einer Quad-Core CPU und Hyperthreading (mit insgesamt 8 ausführbaren Threads) bis zu 8 Container erstellt. Bei 32 GB Arbeitsspeicher könnten dann jedem Container 4 GB zugeteilt werden.\footnote{In der Praxis ist es meistens weniger, da entsprechende Ressourcen für das darunter liegende Betriebssystem und YARN selbst reserviert werden.}\\
Derzeit werden für den Container die Anzahl der CPU-Cores (Ausführbare CPU-Threads) und die Größe des nutzbaren Arbeitsspeichers definiert.\cite[S. 48 ff.]{expert_hadoop_admin}\\

\noindent
Wenn nun eine Anwendung über YARN im Cluster ausgeführt werden soll, dann sendet ein Client eine Anfrage an den Ressourcen Manager. 
Für jeden auszuführenden Job erstellt der Resource Manager den ersten Container. In diesem Container wird dann der Application Manager gestartet, welcher sich dann im weiteren Verlauf um die Ausführung des Jobs kümmert. Der Resource Manager kennt die Anwendung nicht, noch weiß er wie diese ausgeführt werden. 
Er ist nur dafür zuständig Ressourcen zu verteilen.\\ 
Der Application Master hingegen ist sehr spezifisch. Wird zum Beispiel eine Apache Spark Anwendung mit YARN ausgeführt, so ist der Application Master der sogenannte \textit{Spark App Master}. Nachdem der Application Master im ersten erzeugten Container gestartet wurde, kann dieser wiederum neue Ressourcen beim Resource Manager anfordern. 
An dieser Stelle zeigt sich der Vorteil von YARN in Kombination mit dem \gls{hdfs}. Denn bei der Anforderung von Ressourcen gibt der Application Master an, wie viele Container (inklusive Arbeitsspeicher und CPU) er benötigt. Zusätzlich übermittelt er die Dateiblöcke, welche er aus dem \gls{hdfs} braucht und auf welchen Knoten er wie viele Container starten will. So würde der Application Master auf dem Knoten \#2 (siehe Abbildung \ref{fig:yarn_cluster_architecture}) einen Container auf dem Knoten \#2 und zwei Container auf dem Knoten \#4 mit beispielsweise einem GB Arbeitsspeicher und einem Core anfordern. 
Denn der Application Master weiß, dass dort die benötigten Datenblöcke im \gls{hdfs} gespeichert sind. Hierbei ist es wichtig zu verstehen, dass die Knoten aus Abbildung \ref{fig:yarn_cluster_architecture} den Data Nodes aus Abbildung \ref{fig:hdfs_cluster_architecture} entsprechen.\footnote{Wobei ein physikalischer Knoten, auf welchem ein Data Node läuft nicht zwingend auch für die Datenverarbeitung mit YARN verwendet werden muss. Beziehend auf das Paradigma der Datenlokalität ist dies aber der Normalfall, dass ein Knoten, welcher Daten persistiert, auch Daten verarbeiten wird.}\\

\noindent
Der Application Master erhält dann die Zustimmung vom Resoure Manager, nachdem der Scheduler die geforderten Ressourcen entsprechend eingeteilt hat. Darauf fordert der Application Manager die Node Manager auf den jeweiligen Knoten auf, entsprechende Container zu erstellen.\\
Die einzelnen Node Manager stehen in Kontakt zum Resource Manager und senden ihm den aktuellen Status des Knoten und dessen Auslastung. \\
%TODO Nachfolgende Aussage muss nochmals geprüft werden.
Nach der Ausführung der einzelnen Tasks innerhalb der Container und dem Abschluss des Jobs, schickt der Application Master über den Application Manager die Ergebnisse zurück zum Client.\footnote{Hierbei werden die fachlichen Ergebnisse meistens als Datei im \gls{hdfs} gespeichert.} Danach meldet er sich beim Resource Manager. Zuletzt gibt der Resource Manager allozierte Ressourcen frei.\\


\noindent
Ähnlich wie beim Prozessscheduling in ein herkömmlichen Betriebssystem, gibt es auch für YARN unterschiedlicher Algorithmen, die festlegen, in welcher Reihenfolge und Zeitdauer die einzelnen Jobs ausgeführt werden. Bekannte Scheduler sind der \textit{Fair Scheduler} und der \textit{Capacity Scheduler}. Abhängig von der genutzten Plattform/Distribution einzelner Hersteller ist für YARN ein anderer Scheduler konfiguriert. In etlichen Fällen wird der Capacity Scheduler als Standard konfiguriert, da dieser versucht alle Knoten möglichst effizient auszusteuern, um den höchstmöglichen Datendurchsatz zu erreichen. Der Fair-Scheduler prüft hingegen, dass jedem Job die gleichen Ressourcen zugeteilt werden, um möglichst alle Jobs parallel bedienen zu können.\\

\noindent
In großen Clustern wird die Prozessierung in mehrere Sub-Cluster mit eigenen Resource Managern aufgeteilt. Diese Struktur wird in der Literatur als \textit{Federaded YARN} beschrieben und soll die Skalierbarkeit von YARN in großen Clustern ermöglichen.

\section{Apache Spark}
\label{sec:theory_spark}

\textit{Apache Spark\texttrademark\thinspace} ist ein Projekt zur verteilten Verarbeitung von großen Datenmengen. Mit Apache Spark können verschiedene Algorithmen und Verarbeitungsschritte über eine einheitliche Programmierschnittstelle auf gespeicherte Daten  angewendet werden. Spark selbst kümmert sich um die Verteilung, Ausführung und Überwachung der Applikationen zur Datenverarbeitung.\cite[S. 2]{learning_spark}\\
Apache Spark ist mittlerweile schon fast der Standard, wenn es im Hadoop-Umfeld um die Datenverarbeitung geht. Es löst damit auch das ursprünglich verwendete MapReduce-Framework von Hadoop ab, denn Spark
bietet einen enormen Geschwindigkeitsvorteil gegenüber dem MapReduce-Framework. Dies lässt sich auf eine intelligente Ausführung einzelner Verarbeitungsschritte und diverse Optimierungen zurückführen.\cite[S. 148 ff.]{expert_hadoop_admin}\\

\noindent
Spark ermöglicht vielseitige Einsatzzwecke. So wird die klassische Datenverarbeitung von statischen Datenmengen\footnote{In diesem Kontext ist die simple Ausführung einer Anwendung auf eine bereits existierende Datenmenge gemeint, welche am Ende ein definiertes Ergebnis liefert.} unterstützt, aber auch die Verarbeitung von dynamischen Datenmengen (Streaming-Data).\footnote{Bei der Datenverarbeitung von Streaming-Data wächst die zu verarbeitende Datenmenge dynamisch an und die ausgeführte Anwendung verarbeitet die neu hinzugekommenen Daten. Ein Beispiel wäre das Filtern von Tweets auf Twitter nach bestimmten Merkmalen, wobei auch neu hinzukommende Tweets bearbeitet werden und nicht nur die Tweets, welche beim Startzeitpunkt der Anwendung bereits existieren.} Auch die Prozessierung von Graphen-Strukturen und das maschinelle Lernen werden unterstützt.\cite[S. 152]{expert_hadoop_admin}\\

\noindent
Darüber hinaus steht es dem Anwender frei, ob er seine Applikationen in Scala, Python oder Java schreibt. Bei den Interpreter-Sprachen Scala und Python gibt es eine Spark-Shell zur interaktiven Datenverarbeitung und Analyse. Diese Vielseitigkeit macht sich auch in unzähligen Projekten und Programm-Bibliotheken bemerkbar, welche rund um Apache Spark entwickelt werden.
Es existieren diverse Anbindungen zu Datenquellen, die sogenannten \textit{Spark-Connectoren}. Damit können beliebige Datenspeicher als Datenquelle verwendet werden.
Beispielsweise können Daten aus dem \gls{hdfs} geladen werden, aber auch direkt aus Datenbanken wie HBase, Cassandra, Neo4j oder Elasticsearch, welches zur Datenindexierung genutzt werden kann.  \\

\noindent
Abbildung \ref{fig:spark_cluster_architecture} zeigt die Ausführung einer Spark-Applikation innerhalb eines Hadoop-Clusters mit YARN und skizziert den physikalischen Kontext im Cluster. Dieser Aufbau beschreibt im Kontext der Thesis den primären Anwendungsfall zur Datenverarbeitung.\\ 
Apache Spark könnte auch vollständig unabhängig von dem Hadoop-Framework in einem eigenen Spark-Cluster ausgeführt werden und bietet dafür auch einen eigenen Ressourcenmanager. Allerdings wird innerhalb des Hadoop-Umfelds die Spark-Ausführung mit dem bereits erwähnten Ressourcenmanager YARN durchgeführt (siehe Kapitel \ref{sec:theory_yarn}). Dies hat auch
den Vorteil, dass YARN die Ressourcen auf den einzelnen Knoten besser verwalten kann. Denn wenn der Spark-Ressourcenmanager parallel zu YARN auf den gleichen Knoten genutzt werden würde, so könnte dies zu Ressourcen-Engpässen führen. Denn die Ressourcenmanager würden nicht miteinander kommunizieren und die Last der ausgeführten Anwendungen im Cluster könnte nicht gleichmäßig verteilt werden. Aus diesem Grund ist es ratsam YARN auch die Ausführung von Spark-Anwendungen im Cluster zu überlassen.\\

\begin{figure}[ht]
  \centering
  \includegraphics[width=\textwidth]{./resource/spark_cluster_architecture.pdf}
  \caption{Spark Datenverarbeitung im Cluster}
  \label{fig:spark_cluster_architecture}
\end{figure}

\noindent
Wie in Abbildung \ref{fig:spark_cluster_architecture} ersichtlich, wird die Ausführung einer Spark-Anwendung über den YARN-Ressourcenmanager gestartet. Es gibt hierbei unterschiedliche Varianten, wie eine Spark-Anwendung ausgeführt werden kann. 
Im konkreten Fall wird das \textit{Spark-Submit} Kommando genutzt. Letztlich handelt es sich hierbei um einen Konsolenbefehl, welcher die Anwendung\footnote{Beispielsweise ist dies bei einer Spark-Anwendung in Java ein herkömmliches \textit{Java Archiv} im \textit{\gls{jar}}-Dateiformat.} selbst entgegen nimmt und über diverse Parameter konfiguriert werden kann. 
So kann unter anderem der Master und Deploy Mode so konfiguriert werden, dass YARN die Ressourcen der Applikation verwaltet. Wie in Abbildung \ref{fig:yarn_cluster_architecture} (siehe Kapitel \ref{sec:theory_yarn}) bereits beschrieben, wird bei YARN ein Application Master erstellt, welcher wiederum diverse Ausführungscontainer auf den einzelnen Knoten anfordert und diese überwacht. Bei der Ausführung einer Spark-Anwendung werden diese Komponenten wiederverwendet und kapseln letztlich die fachlichen Komponenten von Spark.\\
So gibt es bei Spark einen sogenannten \textit{Driver}, welcher im Yarn Application Master läuft und wiederum die sogenannten \textit{Executor} aussteuert. Diese laufen wiederum gekapselt in einzelnen YARN Containern auf den Knoten. Ein Spark Executor entspricht aus Betriebssystemsicht eines Knotens der Ausführung einer \gls{jvm} in einem eigenständigen Prozess. Aufgrund der Kapselung durch YARN können die einzelnen \gls{jvm}-Prozesse überwacht werden und zur Not auch beendet werden, falls sie zu viel Ressourcen auf den Knoten anfordern.\\

\noindent
Spark und YARN müssen entsprechend konfiguriert werden, damit die Anwendungen auch korrekt im Cluster skaliert werden können. 
%Nachfolgende Beispiel zeigt hier Konfigurationsmöglichkeiten.\\
Bei YARN und auch Spark beziehen sich die Ressourcen auf die Anzahl der genutzten CPU-Cores
%\footnote{Hierbei geht es um die virtuell verfügbaren CPU-Cores. So verfügt beispielsweise eine Intel CPU mit vier physikalischen Cores und Hyperthreading über insgesamt 8 virtuelle Cores zur parallelen Ausführung von Prozessen.}
und die Größe des genutzten Arbeitsspeichers.\\
%TODO Spark + YARN Konfiguration beschreiben. Oder zumindest im Anhang.
%\noindent
%\textbf{TODO}: Configurations Parameter in Cluster!\\
Gerade wenn YARN einzelne Application-Container stoppt, weil sie zu viel Arbeitsspeicher benötigen deutet dies auf eine falsche Konfiguration oder falsche Programmierung der Spark-Anwendungen hin. Um das Problem zu lösen wird gerne der nutzbare Arbeitsspeicher pro Executor höher konfiguriert. Dies ist in den meisten Fällen jedoch der falsche Ansatz, da hierdurch kritische Probleme in der Programmlogik der Anwendung oftmals  nur kaschiert werden.\\ Daher ist es auf jeden Fall auch sinnvoll bei der Anwendungsentwicklung relativ kleine Cluster mit geringen Ressourcen zu nutzen, denn auch dort müssen die Anwendungen fehlerfrei ausführbar sein. Lediglich die Ausführungsgeschwindigkeit sollte sich in kleinen Clustern verlangsamen. Aus diesem Grund werden im Rahmen dieser Thesis auch die Spark-Anwendungen auf einem einzelnen Knoten getestet, um Programmfehler besser und frühzeitiger erkennen zu können. Einzelheiten zu den Programmierparadigmen und den grundlegenden Datenstrukturen können in Kapitel \ref{ch:data_processing} nachgelesen werden.


\section{Apache HBase}
\label{sec:theory_hbase}
\textit{Apache HBase\textsuperscript{\textregistered}} ist eine spaltenorientierte \textit{\acrshort{nosql}}-Datenbank. Sie entstand auf den Grundlagen der \textit{BigTable}-Datenbank von Google und wurde für das Speichern von Daten im Hadoop-Umfeld entwickelt.\footnote{Der Name \textit{HBase} basiert auf der Kombination von \textit{Hadoop} und \textit{Database}.}\\
Der Begriff \textit{NoSQL}-Datenbank steht hierbei für \textit{Not only SQL} und beschreibt letztlich Datenbanken, welche Daten vorwiegend nicht in herkömmlichen relationalen Datenbankschemata speichern. Größtenteils sind diese Datenbanken schemafrei und können horizontal skaliert werden. Diese Bedingungen sind optimal zur Speicherung großer unstrukturierter Datenmengen.\\
Anhand des sogenannten \textit{\acrshort{cap}-Theorems} können diese Datenbanken kategorisiert werden.
Das CAP-Theorem besteht aus den Eigenschaften Konsistenz, Verfügbarkeit und Partitionstoleranz und besagt, dass maximal zwei dieser drei Eigenschaften von einer Datenbank garantiert werden können. Konsistenz beschreibt hier die Garantie, dass alle Knoten im verteilten System den gleichen Datenstand haben. Die Verfügbarkeit bezieht sich auf die dauerhafte Erreichbarkeit der Daten. Wohingegen die Partitionstoleranz die Funktionsfähigkeit bei einem Ausfall einzelner Knoten im Datenbank-Verbundsystem garantiert.
Apache HBase garantiert hierbei die Eigenschaften der Partitionstoleranz und der Konsistenz. Dies führt dazu, dass die Verfügbarkeit der Daten weniger stark ausgeprägt ist.
\cite[S. 189 ff.]{big_data_praxis}\\
% Vielleicht wäre auch hier eine Überlegung die Datenbank Cassandra zu nutzen, welche die Verfügbarkeit und Partitionstoleranz garantiert. An sich genommen ist die Konsistenz der Ergebnisse in der Forensik schon wichtig, allerdings sind nach der ersten Datenalyse eigentlich keine Änderungen auf den existierenden Datensätzen zu erwarten. Vielleicht wäre dann die Konsistenzgarantie ein Stück weit überflüssig und Cassandra könnte bessere Performanz bieten?

\noindent
Während herkömmliche relationale Datenbanken die Daten zeilenweise speichern, werden in spaltenorientierten Datenbanken die Daten spaltenweise gespeichert. Hierbei werden die Daten der einzelnen Spalten gruppiert in sogenannten \textit{Column Families} abgespeichert. Der Vorteil dieser Speicherart hängt stark von deren fachlichen Nutzung ab. Wenn alle Daten einer Spalte abgefragt werden, dann können diese Daten effizienter gelesen werden, da sie zusammen persistiert wurden. Bei einer relationalen Datenbank hingegen wird bei solchen Anfragen die ganze Zeile mit allen Feldern gelesen, obwohl nur ein kleiner Teil dieser Daten wirklich benötigt wird.\\
%\textbf{TODO}: prüfen ob das so wirklich stimmt. Zumindest müssen einzelne Felder übersprungen werden, was ja nach Speicherart auch Zeit kostet.\\

\noindent
Apache HBase basiert eben auf dieser Spaltenorientierung. Die Datenbank wird im Rahmen dieser Thesis für das Speichern beliebiger Dateiinhalte und Dateimetadaten verwendet. Ein vereinfachtes Beispiel einer möglichen Datenstruktur wird in Abbildung \ref{fig:hbase_schema_example} dargestellt.\\

\begin{figure}[ht]
  \centering
  \includegraphics[width=0.8\textwidth]{./resource/hbase_data_schema_example.pdf}
  \caption{Schema-Beispiel einer HBase Tabelle nach \cite{big_data_praxis}}
  \label{fig:hbase_schema_example}
\end{figure}

\noindent
Anhand der Abbildung werden einige Eigenschaften von HBase sichtbar. So existieren zwei \textit{Column Families} mit den Namen \textit{metadata} und \textit{content}. Die Column Family \textit{metadata} enthält wiederum die Spalten \textit{filePath}, \textit{fileSize}und \textit{fileType}. Die Werte werden alle als Binär-Inhalt gespeichert. Eine Zeile hat einen eindeutigen Spaltenschlüssel, wie zum Beispiel \textit{row1}. 
Über diesen Schlüssel können die spezifischen Inhalte der einzelnen Spalten für eine bestimmte Zeile erfragt werden. Interessant hierbei ist, dass die Spaltenwerte optional sind und nicht für jede Zeile existieren müssen. So hat eine Datendatei einen konkreten Inhalt, welcher in der Spalte \textit{fileContent} der Spaltenfamilie \textit{content} gespeichert wird. Ein Verzeichnis hingegen hat keinen Dateiinhalt. 
Daher ist in der zweiten Zeile auch kein Inhalt in der Spalte \textit{fileContent} abgelegt. In der dritten Zeile hingegen, wird ein symbolischer Link gespeichert. Dieser wiederum hat auch keinen Inhalt in der Spalte \textit{fileContent}. Dafür wird aber die Referenz auf die Originaldatei in einer weiteren Spalte gespeichert. Aufgrund der Gruppierung und Speicherung in Column Families benötigen leere Spaltenwerte auch keinen Speicherplatz. 
Ein einzelne Zelle beschreibt letztlich den Wert einer Spalte für eine konkrete Zeile. Hierbei wird zu jeder Zelle auch ein Zeitstempel gespeichert. Durch die Zeitstempel können auch ältere Werte einer Zelle ausgelesen werden. 
So wird bei einer Modifikation einer konkreten Zelle der Wert inklusive eines neuen Zeitstempels geschrieben. Es ist jedoch immer noch möglich, ältere Zustände der Zelle zu lesen.\\
Die einzelnen Spalten innerhalb einer Spaltenfamilie müssen nicht bei der Erstellung einer Tabelle bekannt sein. Lediglich die Spaltenfamilien müssen initial angegeben werden und können später auch nicht mehr geändert werden. Somit kann nachträglich die Tabelle um weitere Spalten erweitert werden.\cite[S. 577]{hadoop_definitive_guide}\\

\noindent
Die Skalierbarkeit und die Partitionstoleranz wurden bei der Entwicklung von HBase berücksichtigt. Es baut auf dem Hadoop \gls{hdfs} auf und speichert darin die Daten. Analog zu \gls{hdfs}, YARN oder Spark existiert auch hierbei eine Master-Slave Architektur über alle Knoten hinweg. Abbildung \ref{fig:hbase_cluster_architecture} zeigt die physikalische Aufteilung.\\

\begin{figure}[ht]
  \centering
  \includegraphics[width=\textwidth]{./resource/hbase_cluster_architecture.pdf}
  \caption{HBase Datenspeicherung im Cluster}
  \label{fig:hbase_cluster_architecture}
\end{figure}

\noindent
Auf den einzelnen Knoten im Computer-Cluster existieren sogenannte \textit{Region Server}. Diese Region Server speichern jeweils unterschiedliche Teile der in HBase angelegten Tabellen. 
Eine Tabelle wird hierbei anhand der Zeilenschlüssel in mehrere Bereiche, den sogenannten \textit{Regions} unterteilt. Beispielsweise könnte die Region A (siehe Abbildung \ref{fig:hbase_cluster_architecture}) alle Daten einer Tabelle der Zeilen 1 bis 3000 enthalten. Die Region B wiederum enthält alle Daten der gleichen Tabelle aber von den Zeilen 3001 bis 8540. Und die Region C könnte die Daten der Zeilen 1 bis 22300 von einer anderen Tabelle enthalten. Somit kann ein Region Server mehrere Regions von gleichen oder auch unterschiedlichen Tabellen verwalten.  Das Prinzip der Datenlokalität greift auch bei HBase. So sollte auf  jedem Knoten, auf dem ein RegionServer läuft, auch ein \gls{hdfs} Data Node existieren. Dieser speichert die Daten der einzelnen Regions als Dateien im \gls{hdfs}.\\

\noindent
Wenn neue Tabellen erstellt werden, oder einzelne Regions zu groß werden, dann koordiniert eine übergeordnete Instanz die Umverteilung von Daten und die Erstellung neuer Regions. Diese Instanz ist bei HBase der sogenannte \textit{HMaster}. Es ist ein leichtgewichtiger Prozess, welcher auf einem beliebigen Knoten im Cluster läuft und über Apache ZooKeeper auch den Status der einzelnen Region Server überwacht.\footnote{Die Funktionsweise von Apache ZooKeeper wird in Kapitel \ref{sec:theory_zookeeper} näher erläutert.} Um die Ausfallsicherheit zu gewährleisten ist auch dieser Prozess redundant ausgelegt.\footnote{Über ZooKeeper kann immer der primäre HMaster-Prozess ermittelt werden. Ist der aktive HMaster nicht mehr erreichbar, schaltet sich sich ein Backup-HMaster ein.} Darüber hinaus kümmert sich der HMaster-Prozess auch um die Restrukturierung bei Teilausfällen einzelner Region Server. 

\noindent
Wenn nun ein Client auf die Daten einer Tabelle zugreifen möchte, muss dieser wissen auf welchem Knoten die Daten abgelegt sind. Hierfür verbindet sich der Client zuerst mit ZooKeeper und erfährt darüber, welcher Region Server die sogenannte \textit{Meta-Tabelle} speichert.\cite[S. 579]{hadoop_definitive_guide}\\
Diese Meta-Tabelle enthält Informationen über alle Tabellen in HBase inklusive der Region-Server und deren Regions die sie bereitstellen. Anhand dieser Metadaten kann der Client dann direkt die benötigten Daten an den entsprechenden Region Servern anfordern.\\
Dieses Vorgehen scheint für den Zugriff auf wenige Daten etwas aufwendig. Es skaliert aber sehr gut bei großen Datenmengen, da kein Flaschenhals vorhanden ist. Normalerweise speichert der Client die angeforderte Meta-Tabelle temporär, so dass er bei nachfolgenden Anfragen direkt auf die entsprechenden Region Server zugreifen kann.\\ 

\noindent
Im Rahmen dieser Thesis wird HBase verwendet, um Metadaten zu speichern. Diese werden wiederum mit Apache Spark ausgelesen und verarbeitet. Hierbei kann das Prinzip der Datenlokalität sehr gut genutzt werden. Denn in der Theorie ist es durchaus möglich die HDFS Data Nodes, die Spark Worker und die HBase Region Server getrennt auf unterschiedlichen Knoten auszuführen. Aber gerade dies macht keinen Sinn, da sonst immer wieder Daten über das Netzwerk transportiert werden müssen und dieses dann zum Flaschenhals der Verarbeitung wird.\\
Sinnvoller ist es HDFS Data Nodes auf den Knoten auszuführen, wo auch die Region Server ausgeführt werden. Darauf aufbauend sollten auch die Spark Worker auf den Knoten gestartet werden, auf welchen die Region Server laufen. Dadurch können im besten Fall die Daten, welche durch Apache Spark benötigt werden, direkt von dem lokalen HBase Region Server bereitgestellt werden. Dieser erhält die Daten wiederum von dem lokalen Data Node. Somit können die Daten direkt auf dem Knoten verarbeitet werden, wo sie auch gespeichert sind und müssen nicht über das Netzwerk an andere Knoten gesendet werden.\footnote{Sie auch Kapitel \ref{sec:theory_yarn} und Kapitel \ref{sec:theory_spark}.} Die Koordinierung dieser Komponenten ist allerdings entsprechend komplex. Hierbei können bei der Entwicklung von Anwendungen viele Fehler gemacht werden, die dieses Prinzip der Datenlokalität aushebeln können. Daher müssen die Implementierungen genau geprüft werden, ob sie das Prinzip der Datenlokalität korrekt umsetzen.\\

\noindent
%TODO : Erstellung eines adäquaten Zeilenschlüssels und dessen Auswirkungen (numerische Sortierung, Hot Spot, ...) -> Dies könnte aber dann in Kapitel 4 oder 5 beschrieben werden!


\section{Apache ZooKeeper}
\label{sec:theory_zookeeper}

Innerhalb eines Computer-Clusters zur verteilten Datenverarbeitung existiert oftmals das Problem, dass sich die einzelnen Komponenten koordinieren müssen. Beispielsweise teilt HBase die Daten auf mehrere RegionServer auf, welche wiederum auf den einzelnen Knoten ausgeführt werden. Doch welche Instanz koordiniert diese Aufteilung? Ein anderes Problem ist der Datenzugriff. Ein Client möchte eine Zeile einer bestimmten Tabelle auslesen. Woher weiß der Client, welchen konkreten Knoten er anfragen muss, um genau diese Zeile zu erhalten?\\ 
In den meisten Fällen existiert hierzu eine bestimmte Instanz, welche die Koordinierung der Knoten übernimmt. Bei HBase gibt es den HMaster, welcher die Koordinierung übernimmt. Das Problem ist hierbei, dass auch der Knoten auf dem diese Master-Instanz läuft ausfallen kann und das komplette System zum erliegen bringt. Diese Verwaltungsinstanzen im Allgemeinen sind kritische Komponenten und können als Single-Point-of-Failure zu einem Stillstand des kompletten Systems führen. Um solche Totalausfälle zur vermeiden, müssen bestimmte Automatismen definiert werden, wie sich die Knoten selbst organisieren können, um einen Ausfall beliebiger Knoten zu überstehen.\\

\noindent
An dieser Stelle bietet \textit{Apache ZooKeeper\texttrademark\thinspace} Mechanismen an, wie sich die einzelnen Knoten in einem verteilten System organisieren und Informationen verteilt synchronisieren können. Aus logischer Sicht stellt ZooKeeper einen Service zu Verfügung. 
Dieser Service ermöglicht das Speichern von Informationen als strukturiertes Verzeichnis mit einzelnen Dateien. Bei den Informationen handelt sich normalerweise um wichtige Konfigurationen, welche im Cluster verteilt werden müssen und zentral über ZooKeeper aktualisiert werden können. Jeder Client, der sich mit ZooKeeper innerhalb des Computer-Clusters verbindet, sieht die gleiche Konfiguration und kann bei Bedarf auch bestimmte Konfigurationen ändern.\cite[S. 4 ff]{professional_hadoop} \\
Aus Sicht des Entwicklers, stellt dieser Service immer die aktuellen Informationen im Cluster zu Verfügung. Wie der Service dies bewerkstelligt ist ein Implementierungsdetail. 
Neben dem bereits erwähnten Konfigurationsmanagement bietet ZooKeeper einen Naming-Service an oder ermöglicht es auch den Live-Status einzelner Knoten zu überwachen.\cite{zookeeper_essentials}\\

\noindent
Wie bei HBase in Kapitel \ref{sec:theory_hbase} bereits erwähnt wird, kann sich ein Client mit ZooKeeper verbinden und erhält über ZooKeeper den aktuellen Knoten, welcher die Metadaten zu allen Tabellen in HBase speichert. Ein anderes Beispiel zeigt die Backup-Instanz des HMaster-Prozesses.
Dieser prüft über ZooKeeper den Status des primären HMaster-Prozesses und wird informiert, wenn letzterer nicht mehr verfügbar ist. Daraufhin propagiert sich die Backup-Instanz als neuen HMaster-Prozess im Cluster, um so die Funktionsfähigkeit von HBase aufrecht zu erhalten. Abbildung \ref{fig:zookeeper_dir} zeigt die Verzeichnisstruktur, welche stark an ein Verzeichnis eines Dateisystems erinnert. Die einzelnen Einträge werden \textit{ZNodes} genannt.\footnote{Siehe Link \url{http://zookeeper.apache.org/doc/r3.5.4-beta/zookeeperOver.html}. Letzter Zugriff: 26.7.2018.} \\

\begin{figure}[ht]
  \centering
  \includegraphics[width=\textwidth]{./resource/zookeeper_dir.png}
  \caption{Gespeicherte Informationen von HBase in ZooKeeper}
  \label{fig:zookeeper_dir}
\end{figure}


\noindent
Normalerweise ist ZooKeeper auf mehreren Knoten im Cluster installiert. Sie bilden ein sogenanntes \textit{ZooKeeper Ensemble} und bestehen zumeist aus 3 oder 5 Einzelinstallationen auf beliebigen Knoten. Ein ZooKeeper Ensemble, welches für die gleiche Anwendungsdomäne zuständig ist, wird auch \textit{Quorum} genannt. Innerhalb eines Quorums gibt es einen Leader und mehrere Follower, die sich die gleichen Konfigurationsinformation teilen. Fällt der Leader aus, können die Follower einen neuen Leader bestimmen.\cite{zookeeper_essentials}\\

\section{Apache Solr und Lucene}
\label{sec:theory_solr}
\textit{Apache Solr\texttrademark\thinspace} ist eine skalierbare und performante Plattform zur Datensuche. Die Daten werden vorher indexiert und können effizient durchsucht werden.\cite{solr_search}\\
Mithilfe von Apache Solr können zum Beispiel eine Volltextsuche für Dokumente oder eine Produktsuche eines Webshops implementiert werden.

\noindent
Im Allgemeinen ist ein Suchsystem in mehrere Analyseschritte aufgeteilt. Ein Großteil dieser Verarbeitungsschritte wird nicht von Solr selbst übernommen. Vielmehr baut Solr auf dem Open-Source Projekt \textit{Apache Lucene\texttrademark\thinspace} auf und erweitert Lucene letztlich um eine skalierbare Infrastruktur und diverse Schnittstellen zur Verarbeitung der Daten.\\
Zu Beginn liegen beliebige Daten in Form von Dateien oder Dokumente vor. Diese Daten durchlaufen eine textuelle Aufbereitung bevor sie indexiert werden können.\\ 
Bei der Textanalyse wird der relevante Text aus den Daten extrahiert. Hierbei hängt die Extraktion der Daten auch davon ab, ob die Daten strukturiert, semistrukturiert oder unstrukturiert sind. Anhand der Dateiformate können diverse Bibliotheken genutzt werden, um Daten zu extrahieren. Beispielsweise kann mithilfe von \textit{Apache PDFBox\textsuperscript{\textregistered}}\footnote{Siehe Link: \url{https://pdfbox.apache.org/}. Letzter Zugriff: 01.08.2018.} Text aus \acrshort{pdf}-Dokumenten extrahiert werden. Mithilfe der \textit{\gls{gdal}}\footnote{Siehe Link: \url{https://www.gdal.org/}. Letzter Zugriff: 01.08.2018.} können Geopositionen aus den \acrshort{exif}-Metadaten von \acrshort{jpeg}-Bildern extrahiert werden. Mithilfe von \textit{Tesseract \acrshort{ocr}}\footnote{Siehe Link: \url{https://github.com/tesseract-ocr/tesseract}. Letzter Zugriff: 01.08.2018.} kann lesbarer Text direkt aus Bildern extrahiert werden.\cite[S. 39]{solr_search}\\

\noindent
Im nächsten Schritt wird der extrahierte Text aufbereitet. Die Aufbereitung hängt stark von dem spezifischen Anwendungsfall und den Daten selbst ab. Im Allgemeinen werden Satzzeichen und überflüssige Füllwörter entfernt und Großbuchstaben werden in Kleinbuchstaben umgewandelt. Darüber hinaus kann auch eine Stammformreduktion durchgeführt werden. Darauf aufbauend kann der Text weitergehend analysiert werden. Beispielsweise könnten Redewendungen erkannt oder Synonyme einzelner Wörter identifiziert werden.\cite[S.44]{solr_search}\\

\noindent
Nach der Aufbereitung des Textes erfolgt die Erstellung eines sogenannten \textit{Inverted Indexes}. Ein \textit{Inverted Index} ist ähnlich aufgebaut, wie ein Stichwortverzeichnis. Jedes Wort wird darin mit den Verweisen zu den Vorkommen in den einzelnen Dokumenten versehen. Auf Basis dieses Indexes können sehr schnell alle Dokumente gefunden werden, welche das gesuchte Wort enthalten.\cite[S. 47]{solr_search}\\

\noindent
Im nächsten Schritt können Suchanfragen auf die erstellten Indizes durchgeführt werden. Aber auch hier gibt es unterschiedliche Modelle, wie die relevanten Dokumente ermittelt werden und in welcher Reihenfolge die Ergebnismenge zurückgeliefert wird.\\
Es gibt das sogenannte \textit{Boolean Model} auf Basis der booleschen Algebra. Beispielsweise könnte eine forensische Suchanfrage lauten: Suche alle Bilder oder Videos, welche das Wort \textit{Unfall} im Dateinamen haben aber nicht kleiner sind als 1 MB. Diese Suchanfrage wird in einen booleschen Ausdruck überführt und die Dateien werden zurückgeliefert. Allerdings liefert dieses Modell keine Aussage über die Reihenfolge der Ergebnismengen. Hierzu gibt es komplexere Suchmodelle, wie das sogenannte \textit{Vector Space Model}. Es basiert auf gewichteten Faktoren und ermöglicht es für jedes Dokument der Ergebnismenge die Trefferwahrscheinlichkeit zu ermitteln. Nach dieser Bewertung wird die Reihenfolge der Ergebnisse ermittelt. Beispielsweise sollte ein Dokument eine höhere Bewertung erhalten, wenn das gesuchte Wort mehrfach im Dokument vorhanden ist.\cite[S. 47 ff]{solr_search}\\

\noindent
Solr besitzt einen \textit{Cloud-Mode}, welcher es ermöglicht die indexierten Daten verteilt auf mehreren Knoten zu speichern. Abbildung \ref{fig:solr_cluster_architecture} skizziert diese Verteilung im Computer-Cluster.\\

\begin{figure}[ht]
  \centering
  \includegraphics[width=\textwidth]{./resource/solr_cluster_architecture.pdf}
  \caption{Solr Cloud-Mode im Cluster}
  \label{fig:solr_cluster_architecture}
\end{figure}

\noindent
Dieser Cloud-Mode wird auch bei der Integration in die Analyseplattform genutzt.\footnote{Siehe Kapitel \ref{ch:data_processing}.} Solr kann als zusätzliches Software-Paket in das Hadoop-Ökosystem integriert werden.\footnote{Zum Beispiel liefert die Hortonworks Dataplatform ein passendes Software-Paket zur Installation von Solr. Primär wird dieses Paket genutzt, um eine performante Suche im \gls{hdfs}-Dateisystem zu ermöglichen. Die Solr-Installation kann aber auch für andere Zwecke genutzt werden.} Es können beliebige Daten in der Solr-Cloud indexiert werden. Darüber hinaus nutzt Solr im Cloud-Mode das bereits beschriebene Zookeeper-Projekt\footnote{Siehe Kapitel \ref{sec:theory_zookeeper}.} zur Koordinierung und Überwachung der einzelnen Knoten.\\
Innerhalb des Cloud-Modes werden die oben beschriebenen Indizes (Inverted Index) in mehrere sogenannte \textit{Shards} aufgeteilt. Ein einzelner Shard ist hierbei ein Teil des Indexes, welcher unabhängig von den anderen Shards durchsucht werden kann. So kann bei einer Suchanfrage die Suche parallel auf alle Shards verteilt werden. Einzelne Shards können wiederum auf anderen Knoten repliziert werden, um die Ausfallsicherheit zu gewährleisten. Solr spricht hierbei von Cores. Ein Core ist die physikalische Repräsentation eines logischen Shards auf einem konkreten Knoten. Alle logischen Shards bilden zusammen den Index, welcher bei Solr als sogenannte \textit{Collection} dargestellt wird.\cite{solr_cloud_scaling}\\

\noindent
Jedes Dokument, welches in einer Collection indexiert werden soll, wird einem der logischen Shards zugeteilt. Hierbei wird zuerst anhand eines eindeutigen Dokumentenschlüssels entschieden in welchen Shard das Dokument aufgenommen werden soll.\footnote{Der Dokumentenschlüssel kann zum Beispiel von der Hashsumme der Datei abgeleitet werden.} Ein logischer Shard wird abermals aufgeteilt in mehrere Replicas, den bereits erwähnten Cores auf den einzeln Knoten. Für jeden logischen Shard wird ein \textit{Leader}-Knoten bestimmt.\footnote{Diese Leader werden mittels ZooKeeper im Cluster propagiert.}. Dieser Leader-Knoten speichert einen physikalischen Core des logischen Shards und indexiert im nächsten Schritt das Dokument. Sobald der Index neu aufgebaut wurde, wird die Aktualisierung an die anderen Knoten weitergesendet, welche wiederum eine Replica (Core) des logischen Shards speichern.\cite[S. 867 ff]{solr_ref_guide}\\ 


\noindent
Apache Solr und Lucene sind beide in Java implementiert. Daher gibt es auch einen Java Client, zur Durchführung von Suchanfragen. Andererseits existiert auch eine \gls{rest}-Endpunkt, der für Anfragen genutzt werden kann. Damit können Anfragen simpel und schnell in beliebige Websites integriert werden oder direkt mit dem Kommandozeilentool \textit{curl} durchgeführt werden. Hierbei können die Anfragen in \acrshort{xml} (\acrlong{xml}) oder \acrshort{json} (\acrlong{json}) formatiert sein. Ein simples Beispiel zeigt nachfolgende Anfrage (siehe Abbildung \ref{fig:solr_request}).\\

\begin{figure}[ht]
  \centering
  \includegraphics[width=\textwidth]{./resource/solr_request.png}
  \caption{Solr-Suchanfrage via \textit{curl}}
  \label{fig:solr_request}
\end{figure}

\noindent
In dem Beispiel ist die Solr-Instanz unter \url{http://localhost:8983} erreichbar. Die Suchanfrage selbst wird auf der Collection \textit{forensicMetadata} ausgeführt. Es soll in allen Feldern nach einem Vorkommen von dem Wort \textit{shot} (\textit{q=*shot*}) gesucht werden. Die Antwort ist im JSON-Format definiert. Unter anderem wird die Anzahl der Treffer (\textit{numFound:1}) mitgeliefert. Bei dieser konkreten Anfrage, wurde ein Bild mit dem Dateinamen \textit{Screenshot...} gefunden. 

\chapter{Datenspeicherung}
\label{ch:data_persistence}

\section{Allgemeines}
Im Praxisteil dieser Arbeit soll eine Analyse-Plattform auf Basis von Hadoop aufgebaut werden. Um die fachlichen Anforderungen an das Analyse-System herauszuarbeiten, soll das herkömmliche Analysevorgehen betrachtet werden.\\
Die Ausgangslage liefert beispielsweise ein Testszenario, welches ein Datenträgerabbild eines virtualisierten Linux-PCs enthält. Das Abbild selbst ist 10 GB groß.\\

\noindent
Dieses Abbild soll als Referenz für die einzelnen Analyse-Schritte gelten. 
Hierbei sollen in jedem der nachfolgenden Unterkapitel die fachlichen Verarbeitungsschritte auf herkömmlichen Wege mit der hier erstellten Analyse-Plattform verglichen werden.\\
Ein gängiges Analysevorgehen besteht aus der Speicherung des Abbildes auf dem lokalen Analyse-Rechner und dem Analysieren des Beweismittels mit Betriebssystemprogrammen unter Linux oder mithilfe des Open-Source Analysetools \textit{Autopsy}\footnote{Siehe \url{https://www.sleuthkit.org/autopsy/}. Hierbei wird die Version 4.6.0 in der 64-bit Variante unter Windows 10 Pro genutzt.} unter Windows. Im kommerziellen Bereich existieren noch deutlich mehr Analyse-Tools mit größerem Funktionsumfang. 

\section{Traditionelles Analysevorgehen}
In einem traditionellen Analysefall wäre als Beweismittel ein Datenträgerabbild des verdächtigen Computers erstellt worden. Mithilfe klassischer Opensource-Tools wie beispielsweise \textit{dd} kann damit zum Beispiel ein Image im RAW-Format erstellt werden. Andere Tools, wie beispielsweise \textit{FTK-Imager} können auch Datenträgerabbilder in speziellen Container-Formaten erstellen und lesen. Beispielsweise gibt es das \textit{EnCase Physical}-Format mit der Dateiendung \textit{e01}, oder das \textit{Advanced Forensic Format} mit der Endung \textit{.aff}.\footnote{Siehe \url{https://support.accessdata.com/hc/en-us/articles/222778608-What-Image-Formats-Do-AccessData-Products-Support-}, Stand: 4.4.2018.}.\\

\noindent
Einige Tools nutzen spezielle Container-Formate. Teilweise wird auch unterschieden, ob es sich um ein vollständiges Disk-Image handelt oder um ein logisches Dateiarchiv. Bei dem vollständigen Disk-Image werden auch nicht allokierte Speicherbereiche innerhalb des Dateisystems, der Partition oder des Datenträgers gesichert. Hier können sich potentiell versteckte und gelöschte Dateifragmente befinden. Mit diversen Tools kann ein sogenanntes File-Carving durchgeführt werden. Hierbei wird nur anhand der Rohdaten im Abbild versucht, Daten zu extrahieren. Die ist gerade auch dann sinnvoll, wenn nach gelöschten Dateien gesucht wird oder die Dateisystemmetadaten irreparabel beschädigt wurden.
\\ Ein logisches Dateiarchiv hingegen enthält wirklich nur die Dateien auf einer logischen Ebene und keine unallokierten Speicherbereiche. Von Vorteil hierbei ist eine geringere Speichergröße. Allerdings tritt durch die logische Sicherung ein potentieller Informationsverlust auf, da unallokierte Speicherbereiche nicht berücksichtigt werden, die aber dennoch potentiell auswertbare Informationen liefern könnten. Vertreter logischer Darteiarchive sind die allseits bekannten Archiv-Formate wie beispielsweise ZIP oder TAR.\\

\noindent
Letztlich werden die Beweismittel in unterschiedlichsten Formaten auf dem lokalen Analyse-Rechner gespeichert. Darauf aufbauend können die Daten in spezifische Formate konvertiert werden. Dies hängt aber meistens davon ab, wie sie weiter verarbeitet werden sollen und welche Werkzeuge zu dieser Verarbeitung genutzt werden.\\

\noindent
Im konkreten Testszenario ist das Datenträger-Abbild eines Linux-Rechners im RAW-Format auf dem lokalen Analyse-Rechner gespeichert. Das Abbild selbst kann ein oder mehrere Partitionen enthalten. Innerhalb der Partition werden Daten mithilfe unterschiedlicher Dateisysteme strukturiert gespeichert. Diese Dateisysteme können unter Windows mit dem Werkzeug \textit{X-Mount} oder unter Linux direkt mit dem Befehl \textit{mount} schreibgeschützt gemountet werden. Darauf wird das Dateisystem vom Betriebssystem interpretiert und als logisches Volume auf dem Analyse-Rechner bereitgestellt. Nun können die Dateien mit beliebigen Werkzeugen analysiert analysiert werden.\\

\noindent
In der Praxis hat das einfache schreibgeschützte Mounten den Vorteil, dass der Analyst relativ schnell im Dateisystem beliebige Dateien finden und dessen Inhalt mit diversen Tools anzeigen kann. Gerade für eine schnelle Vorprüfung ist dies sinnvoll. Im nachfolgenden Kapitel sollen nun Möglichkeiten zur Speicherung und Aufbereitung des Datenträgerabbildes mithilfe der forensischen Analyse-Plattform untersucht werden.


\section{Umsetzung in der Hadoop Analyse-Plattform}

Die Datenspeicherung im HDFS ist ein wichtiger Aspekt in dieser Thesis. Hierbei geht es nicht nur darum, wie die Daten im Hadoop-Framework verwaltet werden, sondern vielmehr um die Art und Weise, wie Daten forensisch korrekt gespeichert werden können.\\

\noindent
Der Standardfall wäre hierbei ein Rohdatenträgerabbild (RAW-Image), welches das Beweismittel darstellt. Hierbei liegen die Dateien als fragmentierte Blöcke in einer spezifischen Datenstruktur vor, welche das Dateisystem des Abbildes beschreiben. Je nachdem, ob ein Datenträgerabbild direkt von einer Partition eines Datenträgers oder vom ganzen Datenträger erstellt wurde, sind in dem Abbild unter Umständen auch mehrere Partitionen samt Partiationstabelle enthalten. Von diesen Partition kann wiederum jede einzelne Partition ein eigenes Dateisystem, wie beispiels FAT, ext4, oder NTFS enthalten. Dieses Dateisystem enthält dann die eigentlichen Dateien, welche logisch zusammengesetzt werden müssen.\\ Viele Linux-Distributionen bieten hier bereits eine weitreichende Unterstützung zum Lesen und Schreiben dieser Dateisysteme. Hierzu können die Dateisysteme einzelner Partitionen des Datenträgerabbildes \textit{gemountet} werden.\\

\noindent
Zur Speicherung der Daten im Hadoop-Framwork gibt es mehrere Möglichkeiten, deren Vor- und Nachteile nachfolgend dargestellt werden sollen.

\subsection{Variante 1 - Datenträgerabbild im HDFS speichern}

Die naheliegende Variante zur Speicherung der Beweismittel, wäre die Datenträgerabbilder direkt im HDFS abzuspeichern. Allein die Größe der Abbilder wäre nicht problematisch. Um eine entsprechende Aufteilung kümmert sich das HDFS.\\ Allerdings hat die Lösung den entscheidenden Nachteil, bei der Weiterverarbeitung der Daten. Auf Betriebssystemebene lassen sich solche Datenträger einfach mounten. Zumal die meisten Betriebssysteme auch ein großen Teil der unterschiedlichen Dateisysteme auf einem Datenträger erkennen können. Aber innerhalb des Hadoop-Frameworks findet sich keine Unterstützung zum Lesen von beliebigen Dateisystemen. Denn normalerweise nutzen JAVA-Applikationen ein definierte Schnittstelle auf Basis von Dateien, die wiederum vom Betriebssystem bereitgestellt werden. Um die logischen Dateien aus dem Datenträgerabbild extrahieren zu können, müsste für jedes einzelnes Dateisystem eine eigene Implementierung geschrieben werden.\\ 

\noindent
Darüber hinaus wäre das Extrahieren der Dateien aus einem Dateisystem auf einem Datenträgerabbild auch nicht wirklich performant. Angenommen das Auslesen würde mit Apache Spark durchgeführt werden. Aufgrund der eingangs beschriebenen Datenlokalität (siehe Kapitel \ref{ch:theory_hadoop}) wären auf jedem Knoten einzelne Blöcke von beispielsweise 128 MB Größe vorhanden. Um dann im ext4-Dateisystem eine Datei lesen zu können, sollten zumindest die Dateisystemmetadaten verfügbar sein. Darüber hinaus können die Fragmente einer einzelnen Datei verstreut innerhalb des Dateisystems liegen. Je nach Grad der Fragmentierung des Dateisystems, müsste dann auf einem Data-Node innerhalb des Clusters schlimmstenfalls dutzende weitere Blöcke anderer Knoten nachgeladen werden, um den Inhalt einer einzelnen Datei zu verarbeiten. Dies würde das Prinzip der Datenlokalität aushebeln. Abbildung \ref{fig:ext4_to_hdfs} skizziert diese verstreute Aufteilung einer Datei im physikalischen Hadoop Cluster.\\ 
% Abbildung TODO: Inode #23 enthält nur Metadaten! Nicht die ganze Date ist ein Inode.
\begin{figure}[ht]
  \centering
  \includegraphics[width=\textwidth]{./resource/ext4_to_hdfs.pdf}
  \caption{Aufteilung der Daten des Datenträgerabbildes im Hadoop-Cluster}
  \label{fig:ext4_to_hdfs}
\end{figure}

\noindent
Das Abbild zeigt eine Datei, welche aus logischer Sicht einen Dateinamen, Metadaten (beispielsweise Zugriffsrechte) und einen Inhalt besitzt. Der Inhalt ist in 9 Blöcken zu jeweils 2048 Byte aufgeteilt.\footnote{Die Blockgröße wird hierbei vom Ext4-Dateisystem bestimmt und ist die kleinste allozierbare Einheit im Dateisystem. In der physikalischen Aufteilung im HDFS Dateisystem gibt es auch Blöcke, welche aber beispielsweise eine Größe von 128 MB aufweisen (orange in Abbildung \ref{fig:ext4_to_hdfs}).} Um nun den Inhalt aus dem ext4-Dateisystem einer Datei auszulesen. Wird zuerst der Superblock benötigt. Dieser enthält allgemeine Informationen zum Dateisystem. Darauf wird die Gruppendeskriptortabelle benötigt, um auf einzelne Blockgruppen zuzugreifen.\footnote{Das Dateisystem selbst in mehrere autarke Bereiche unterteilt, welche für sich genommen eigenständig Daten einer Teilmenge aller Dateien vorhalten. Dies sind sogenannten Blockgruppen.} Über eine Blockgruppe kann wiederum auf die Inode-Tabelle zugegriffen werden. Diese speichert die Metadaten einzelner Dateien als sogenannte Inodes ab. Ein konkreter Inode-Eintrag hält wiederum Verweise auf die konkreten Blöcke, welche den Dateiinhalt beschreiben. Der Dateiname selbst wiederum in den logisch übergeordneten Verzeichnissen gespeichert. Das oberste Verzeichnis, ist das Wurzelverzeichnis. Dieses enthält den Namen einer Datei und die Referenz zum Inode.\\

\noindent
Letztlich zeigt diese Struktur, dass selbst beim Auslesen einer kleinen Datei etliche Speicherstellen innerhalb des Dateisystems gelesen und interpretiert werden müssen. Angenommen, dass ein konkretes ext4-Dateisystem in einer 100 GB großen Partition gespeichert wird, so wir diese große Datei im HDFS-Dateisystem in 800 große Blöcke zu je 128 MB aufgeteilt und auf den einzelnen Knoten des Clusters gespeichert. Hierbei kann aber kein Einfluss darauf genommen werden, wo welche Blöcke mit welchem Inhalt gespeichert werden. Letztlich bedeutet dies wiederum, wenn in einem Apache Spark Executor zu Verarbeitung der Daten eine Datei des Knotens gelesen werden soll, müssen prinzipiell etlichen Datenbereiche von anderen Knoten angefordert werden. Und dadurch wird auf Netzwerkebene unnötig viel Last erzeugt. Letztlich gilt für das Prinzip der Datenlokalität, dass die einzelnen Blöcke im HDFS möglichst unabhängig von einander verarbeitet werden können. Diese Problematik trifft übrigens nicht nur bei Ext-Dateisystemfamilie auf sondern auch bei anderen Dateisystemen.\\

\noindent
Aus den oben genannten Gründen ist die direkte Speicherung kompletter Datenträger im HDFS nicht geeignet für die Analyse im Hadoop-Cluster.

\subsection{Variante 2 - Logische Dateien im HDFS speichern}
Die zweite Variante wäre das Beweismittel auf dem lokalen Analyserechner zu mounten und alle Dateien auf logischer Ebene direkt in das HDFS zu importieren. 
Damit wäre die gesamte Dateisystemstruktur aus dem Datenträgerabbild im HDFS abgelegt. Allerdings wäre File Carving beziehungsweise das Auffinden von gelöschten Dateien nicht mehr im Hadoop-Framework möglich, sofern nicht auch der das Datenträgerabbild selbst nochmals in das HDFS importiert wird.\\ 

\noindent
Interessanter an dieser Variante ist aber das Verhalten des HDFS bezüglich der Metadaten und der unterschiedlichen Dateigröße. Bei dem Importierten müsste darauf geachtet werden, dass alle Metadaten des lokalen Dateisystem im Datenträgerabbild unverändert in das HDFS kopiert werden. 
Welche Metadaten bereits im HDFS mit angelegt werden, zeigt Abbildung \ref{fig:hdfs_file_properties} anhand eines Ausschnitts aus der Web-Repräsentation eines HDFS.\\
\begin{figure}[ht]
  \centering
  \includegraphics[width=\textwidth]{./resource/HDFS_FS_Example.png}
  \caption{HDFS - Dateieigenschaften}
  \label{fig:hdfs_file_properties}
\end{figure}

\noindent
Daraus ist ersichtlich, dass jede Datei entsprechende Dateirechte hat und einem Nutzer und einer Gruppe zugeordnet ist. Zusätzlich wird die Größe (nicht bei Verzeichnissen) und der Zeitstempel der letzten Änderung gespeichert. Die Anzahl der Replikationen und die Blockgröße sind spezifisch für das HDFS. Jede Datei kann auf einer unterschiedlichen Anzahl von Knoten repliziert sein. Die Standardkonfiguration definiert 3 Replikationen im realen Cluster, wobei Verzeichnisse nur logisch auf dem Name Node gehalten werden und damit auf keinem DataNode explizit repliziert werden. Auch die Blockgröße ist in der Standardkonfiguration auf 128 MB festgelegt. Wie im Grundlagenkapitel erwähnt, werden die Dateien in mehren Blöcken zu maximal 128 MB (konfigurationsabhängig) gespeichert und auch repliziert.\\ Bezogen auf klassische Dateisystem entspricht dies auch einem Block im ext4-Dateisystem oder einem Cluster im NTFS-Dateisystem (TODO: Quellenangabe!!! S.22 in M111). Es ist letztendlich die kleinste allozierbare Dateneinheit im Dateisystem. Dies bedeutet allerdings nicht, dass für jede Datei im HDFS auf den jeweiligen Date Nodes immer mindestens 128 MB Speicher belegt werden. Denn die reale Speicherbelegung auf dem Data Node beschränkt sich auch auf die reale Größe der Datei im lokalen Dateisystem des Data Nodes.\\

\noindent
Mit dem Befehl aus Listing \ref{lst:hdfs_put_command} können Dateien von einem lokalen Verzeichnis in das HDFS importiert werden. 
\begin{lstlisting}[label={lst:hdfs_put_command},caption= Befehl zum Speichern einer Datei im HDFS,captionpos=b,frame=single,style=customshell]
hdfs dfs -put test.pdf /test.pdf
\end{lstlisting}
Hierbei werden die ursprünglichen Metadaten der Datei nicht übernommen. So beschreibt der oben erwähnte Modifikationszeitstempel den Zeitpunkt, zu dem die Datei im HDFS das letzte Mal geändert wurde. Dies entspricht initial dem Import-Zeitpunkt. Auch werden Nutzer und Gruppenrechte nicht übernommen. Prinzipiell wäre es aber möglich die Metadaten aus dem lokalen Dateisystem mit in das HDFS zu übernehmen.\footnote{Hierzu kann dem \textit{put}-Befehl aus Listing \ref{lst:hdfs_put_command} der Parameter \textit{-p} mit übergeben werden.} Ratsam ist dies jedoch nicht, da gerade auch der Nutzer, die Gruppe und die dazugehörigen Zugriffsrechte in einem produktiven HDFS-Cluster verwendet werden, um Zugriffsbeschränkungen einzelner Nutzer und einzelner Programme umzusetzen.\\

\noindent
In der Theorie können noch weitere Metadaten gespeichert werden. Hierbei können erweiterte Dateiattribute beliebige Informationen über die Datei speichern. Mit nachfolgendenen Befehlen kann beispielsweise der originale Zeitstempel der Erstellung einer Datei hinzugefügt und ausgelesen werden. Hierbei kann der Name des Attributs und dessen Inhalt frei gewählt werden.
\begin{lstlisting}[label={lst:hdfs_fattr_command},caption= Befehl zum Hinzufügen und Auslesen von Metadaten,captionpos=b,frame=single,style=customshell]
# Create custom file attribute
hdfs dfs -setfattr -n user.ntfs.creationtime -v "2018-04-07T11:14:42,798583789+02:00" /test.pdf

# Read custom file attribute
hdfs dfs -getfattr -d -n user.ntfs.creationtime /test.pdf
\end{lstlisting}

\noindent
Mit dem obigen Befehl wäre es also prinzipiell möglich, Metadaten des ursprünglichen Dateisystems zu übernehmen. Allerdings müssen diese Metadaten zur Weiterverarbeitung zuerst für jede Datei im HDFS eingetragen werden.\\
Zusätzlich müssen beim Prozessieren der Daten mit Apache Spark die Metadaten auslesbar sein. Dieses Auslesen ist umständlich aber möglich.\footnote{Siehe Metadaten-Extraktion im Projekt \textit{foam-processing-spark} unter \url{https://github.com/jobusam/foam-processing-spark}.}\\ 
Letztlich ist es wohl aber auch ein Performance-Problem, denn diese Metadaten werden auf dem Namenode gespeichert.\footnote{Siehe \url{https://de.hortonworks.com/blog/hdfs-metadata-directories-explained/}, Stand 6.4.2018.} Somit müsste jeder Datanode diese Informationen wiederum explizit am Namenode anfragen.\\

\noindent
Das Fazit der Variante 2 lautet daher, dass die Dateimetadaten des originalen Datenträgerabbildes höchstens als erweiterte Attribute im HDFS abgelegt werden könnten. Beim Importieren müssten diese Metadaten bei jeder einzelnen Datei explizit nachgetragen werden. Pro Datei wäre dies nochmals ein eigener HDFS-Aufruf. Nicht zuletzt werden diese erweiterten Metadaten physikalisch im Namenode gespeichert. Dadurch benötigt der Namenode mehr Speicher. Viel schwerwiegender jedoch ist, dass die erweiterten Metadaten beim Prozessieren mit Spark angefordert werden müssten und somit der Namenode viel zu stark beansprucht werden würde.\\
Aus diesen Gründen ist die Variante 2 nicht akzeptabel. 

\noindent
TODO: weiterer Aspekt - performantes Prozessieren von vielen kleinen Dateien / bwz. Dateien unterschiedlicher Größe!!!\\

\subsection{Variante 3 - ZIP-File bzw. Hadoop Archive?}
Die Nachteile aus Variante 2 beziehen sich in erster Linie auf die Speicherung der Datei-Metadaten. Diese sollten nicht als erweiterte Dateiattribute im HDFS gespeichert werden, da sie sonst bei der Verarbeitung den Name Node im Hadoop Cluster überfordern könnten. Ein weiterer Nachteil wäre hierbei auch, dass das HDFS sehr gut mit großen Dateien umgehen kann, aber für kleine Dateien ( << 128 MB) nicht ausgelegt ist. Weil eben die Metadaten im Name Node gespeichert werden und generell der Ressourcenaufwand der Speicherung einer HDFS-Datei nicht im Verhältnis zu einer kleinen 1 kByte-Datei steht. Dies zeigt sich auch bei den Testdaten. Auf einem 10 GB großen Testdatenträgerabbild wurde ein Ubuntu-Linux installiert und einige Nutzeroperationen durchgeführt. Nachfolgende Abbildung zeigt die Anzahl der Dateien gruppiert nach deren Größe. Hier wird deutlich, dass es bei der Analyse von Datenträgern primär um das Verarbeiten sehr vieler und sehr kleiner Dateien geht. 

Costum Forensic Format
Metadaten,Blockaufteilung,logische Sicherung\\
nachteil, carving muss vorher passieren...

Nachteil, wenn Metadaten (Hashes,File-Typen, etc...) hinzukommen müssten die Dateien neu berechnet werden...

\subsection{Variante 4 - Speicherung mit HBASE und HDFS}
Diese Variante beschreibt einen Ansatz, bei welchem die Metadaten und sehr kleine Dateien direkt in HBASE persistiert werden. Dateien größer 10 MB werden direkt im HDFS abgelegt und im jeweiligen Dateieintrag in HBASE verlinkt. Auch hier muss beim Import der Daten eine eigene Anwendung die Informationen aus dem Datenträgerabbild auslesen und entsprechend ihrer Struktur in HBASE und HDFS speichern. Das GitHub-Projekt \textit{foam-data-import} enthält diese Anwendung (siehe \url{https://github.com/jobusam/foam-data-import}).\\


Abbildung \ref{fig:data_import} skizziert die Datenaufbereitung und Speicherung in HBASE und im HDFS.\\
\begin{figure}[ht]
  \centering
  \includegraphics[width=\textwidth]{./resource/datenimport_hadoop.png}
  \caption{Datenimport in HBASE und HDFS}
  \label{fig:data_import}
\end{figure}

\subsubsection{Datenmodell}

Das Datenmodell beschreibt die Struktur der Metadaten und verdeutlicht den Informationsgehalt der forensischen Analyseplattform. 

\begin{table}[!ht]
\centering
\begin{tabular}{|l|l|p{8cm}|}\hline
\bfseries Spaltenfamilie & \bfseries Spaltenname & \bfseries Beschreibung
\csvreader[head to column names]{resource/hbase_datamodel.csv}{}
{\\\hline \columnfamily & \column & \description}\\\hline
\end{tabular}
\caption{Datenmodell in HBASE}
\label{tab:hbase_datamodel}
\end{table}

TODO: Datenimport mit unterschiedlicher Dateigröße aufzeigen.

%\noindent
%TODO: Siehe HBASE MOB (Medium-sized Object Binaries)
TODO: Fazit der vor und nachteile aller varianten, um zu zeigen wie variante 2 besser als 1 ist und variante 3 besser als 2 wäre und variante 4 die beste variante ist!
\section{Fachliche Probleme bei den Daten}

\subsection{Symbolische Links}
Es gibt einige Hürden beim Importieren der Datenträger in das HDFS. Wie bereits erwähnt werden die Daten auf logischer Dateiebene in das Hadoop-System importiert. Hierbei müssen spezielle Dateitypen berücksichtigt werden. Ein Beispiel ist die Verarbeitung von symbolischen Links, welche gerade unter Linux-basierten Betriebssystemen beziehungsweise in der EXT-Dateisystemfamilie auftreten können. Denn wenn symbolische Links in einem Dateisystem gespeichert werden und letzteres im Analysesystem gemountet wird, so können diese Links auch auf Dateien außerhalb des Dateisystems verweisen. Denn letztlich interpretiert das Betriebssystem diese symbolischen Links. Bei der forensischen Analyse könnte diese aber zu schwerwiegenden Fehlern der Analyseergebnisse führen, wenn beispielsweise Inhalte des Analyserechners verarbeitet werden, welche ursprünglich nicht auf dem Asservat vorhanden waren. Daher muss beim Import auf geprüft werden, ob die Datei einem symbolischen Link entspricht. Ist dies der Fall darf, der symbolische Link nicht interpretiert werden. 

\subsection{Zugriffsrechte}
Ein weiterer Aspekt ist die Beschränkung der Dateizugriffe auf Basis der vorgegebenen Zugriffsrechte. Angenommen es wird ein Datenträgerabbild auf einem Analysesystem gemountet.
Dann kann dies im Normalfall nur mit Root-Rechten durchgeführt werden. 
Der forensische Analyse benötigt also zumindest auf seinem Analyse-Rechner 
privilegierte Ausführungsrechte.\\
Beim Import von Dateien auf eben diesem gemounteten Dateisystem des Datenträgerabbildes sind jedoch die Dateizugriffsrechte weitaus interessanter. Denn das Analyse-Betriebssystem berücksichtigt diese Zugriffsrechte. Während diese Problematik beispielsweise bei NTFS-Dateisystemen eine untergeordnete Rolle spielt, so werden hingegen bei EXT-Dateisystemen die Unix-Dateirechte gespeichert und auch auf dem Analysesystem interpretiert. 
Daher kann der Nutzer und dessen ausgeführte Programme, welche die Daten aus dem Dateisystem auslesen, nicht in allen Fällen auf alle Dateien zugreifen.\\

\noindent
Die einfachste Möglichkeit um die Problematik der Zugriffsrechte zu umgehen, wäre das Ausführen der Datenimport-Applikation mit Root-Rechten. Andererseits sollte die Datenimport-Applikation aber nicht mit Root-Rechten ausgestattet werden, da dies zu Sicherheitslücken und unvorhergesehenen Rechteausweitungen im Fehlerfall führen könnte. Darüber hinaus kann bei einem Fehlverhalten der Anwendung das Analysesystem beschädigt werden. \\
Letztlich braucht die Anwendung zum Datenimport nur die Berechtigungen zum Lesen von Dateien mit Root-Rechten. Dies würde dem \textit{Need-to-know-Prinzip} genügen.\\
%(TODO: verweis auf s. 42 M106)

\noindent
Eigentlich müsste beim Mounten des Dateisystems dem Betriebssystem mitgeteilt werden können, dass die Dateirechte des gemounteten Dateisystems ignoriert werden sollen. Diese Option ist aber nicht möglich\footnote{Zumindest konnte keine funktionierende Variante gefunden werden (siehe Man-Page des Mount-Befehls.}.\\
Eine weitere Alternative wäre die Möglichkeit mit Access Control Lists (ACL) zu arbeiten und dem nichtprivilegiertem Nutzer Rechte zum Lesen der Dateien zu geben. Oder umgekehrt alle Dateien dem nichtprivilegierten Nutzer zuzordnen, welcher wiederum den Datenimport startet.
Hierzu müsste die Datenträgerkopie schreibend gemountet werden, damit die Rechte jeder Datei angepasst werden können. Dies würde wiederum dazu führen, dass das Datenträgerabbild als sichergestelltes Asservat geändert werden würde. 
Daher ist diese Lösung auch nicht geeignet.\\

\noindent
Eine weitere Alternative ist die Nutzung von Posix Capabilities\footnote{Siehe Manpages mit folgendem Befehl: \textit{ man 7 capabilities}.}.
Dies scheint unter CentOS/Fedora wohl die beste Variante zu sein. Zum Lesenden Zugriff auf Dateien müsste die Posix Capability \textit{CAP\_DAC\_READ\_SEARCH} gesetzt werden.\\
Mit nachfolgenden Kommando könnte diese Capability dem Analyseprogramm gesetzt werden.
Damit könnte dann auch ein nicht-privilegierte Nutzer lesenden Zugriff auf privilegierte Dateien erhalten.\\ 

\begin{lstlisting}[label={lst:pos_cap_command},caption= Befehl zum Setzen von Posix Capabilities,captionpos=b,frame=single,style=customshell]
setcap CAP_DAC_READ_SEARCH /bin/forensic_import
\end{lstlisting}
% Mit oder ohne + eip???
%setcap CAP\_DAC\_READ\_SEARCH+eip /bin/ping

\noindent
Allerdings funktioniert diese Art hauptsächlich bei Binärprogrammen aber nicht bei Shell-Skripten oder Java-Anwendungen. Ein ähnliche Alternative zu Posix Capabilities ist das Setzen des SUID-Bits als Unix-Dateirecht für die Programmdatei. Aber auch gilt wieder, dass dies für Binärprogramme funktioniert aber nicht für interpretierte Skripte oder Java-Anwendungen, welche wiederum in der Java Virtual Machine laufen.
%Setzen des Flags
%chmod u-s /bin/ping

Die Nutzung von SELINUX empfiehlt sich hier auch nicht. SELINUX ist eine Erweiterung des Linux-Kernels um detaillierte Zugriffskontrollen zu setzen. Diese Erweiterung ermöglicht eine detaillierte Definition, welche Anwendung oder Nutzer auf einzelne Dateien zugreifen dürfen. Allerdings werden auch unter SELINUX immer zuerst die Unix-Dateirechte und zusätzlich die in SELINUX definierten Zugriffskontrollen geprüft. Daher kann auch mit SELINUX das Problem nicht behoben werden.\\

\noindent
Zuletzt gibt es noch eine Variante, welche die Problematik mit den Dateirechten lösen kann. 
Mit dem Projekt \textit{bindfs}\footnote{Siehe \url{https://bindfs.org/}.} können unter Linux Dateisystemverzeichnisse neu gemountet werden und ihre Zugriffsrechte verändert werden. Der nachfolgende Befehl mountet das existierende Verzeichnis mit den enthaltenden Dateien in einem neuen Verzeichnis und setzt beispielsweise bei jeder Datei die aktuelle ID des Nutzers als Datei-Owner und Group. 
\begin{lstlisting}[label={lst:bindfs_command},caption= Nutzung von Bindfs zum Ändern von Dateirechten,captionpos=b,frame=single,style=customshell]
sudo bindfs -u $(id -u) -g $(id -g) src_dir/ target_dir/
\end{lstlisting}
Der Befehl selbst benötigt Root-Rechte. Jedoch kann der Nutzer danach alle Dateien des Zielverzeichnisses lesen. Der einzige Nachteil an dieser Lösung ist, dass der Besitzer und die Gruppe jeder einzelnen Datei nun der Nutzer des Analysesystems ist.\\
Dieser Nachteil muss zukünftig behoben werden, damit die forensische Analyseplattform auch die Besitzer und Gruppen einer Datei korrekt auswerten kann.
%Im Anhang beschreiben, wie bindfs gebaut werden kann.
\chapter{Datenverarbeitung}
\label{ch:data_processing}

\section{Verarbeitung Apache Spark\texttrademark}
Der physikalische Aufbau wurde bereits im Grundlagenkapitel zu Apach Spark behandelt (siehe Kapitel \ref{sec:theory_spark}). In diesem Kapitel sollen primär die Algorithmen und die Verarbeitung der Daten aus logischer Sicht betrachtet.\\ 

\noindent
Bei Apache Spark gibt es seit der Version 2.0.0 diverse APIs, wie Daten geladen werden können. Es besteht die Möglichkeit Daten mithilfe von Resilient Distributed Datasets (RDDs) zu laden und zu verarbeiten. Aufbauend auf diesen RDDs können, die Daten gemappt, gefilter oder aggregiert werden. Dies Möglichkeit gibt es schon immer in Apache Spark. Seit der Version 2.0.0 gibt es nun auch DataFrames und DataSets (TODO: gibt es beides erst seit v2.0?). Diese Datenstrukturen beschreiben eher eine Schnittstelle aus Sicht von Tabellen. Die Implementierung dieser Typen baut wieder auf den RDDs auf. Doch welche Strukturen eignen sich für die Anwendungsfälle in dieser Thesis? \\

\noindent
DataFrames und DateSets sind optimiert für strukturierte und semi-strukturierte Daten. Diese Daten lassen sich beispielsweise in Tabellenstrukturen einlesen und verarbeiten. Es gibt High-Level Operationen auf diesen Tabellen, welche dem klassischen SQL Syntax sehr nahe kommen?! Apache Spark selbst kann bei der Nutzung von DataFrames und DataSets viele Optimierungen bei der Ausführung und Verarbeitung durchführen. Andererseits sind diese Strukturen ungeeignet bei unstrukturierten Daten, wie beispielsweise Multimediadateien und eben auch beliebigen Dateien.\cite[S. 66 ff.]{data_processing_spark2}\\

\noindent
Wie beim Datenimport schon beschrieben, sind die Metadaten der analysierten Datenträger strukturiert beziehungsweise semi-strukturiert in HBASE abgespeichert. Prinzipiell wäre es also möglich, auch mit Datasets und Dateframes auf diese Daten zuzugreifen. Letztlich kommt es auch auf die Anbindung zwischen Apache Spark und Apache HBASE an. Hierbei gibt es primär zwei unterschiedliche Connectoren\footnote{Bei Apache Spark sind Connectoren eine Art von Java-Bibliotheken, welche es ermöglichen im Apache Spark Ausführungskontext auf andere Systeme, wie beispielsweise Datenbanken oder Dateisysteme, zuzugreifen.}. Der \textit{Hortonworks SHC} Connector ermöglicht die Interaktion mit Daten in HBASE und nutzt dafür die DataFrame/DataSet Datenstrukturen.\footnote{Siehe \url{https://github.com/hortonworks-spark/shc}, Stand: 15.6.2018.} 
Also Pendant auf Basis von RDDs existiert ein weiterer \textit{hbase-spark} Connector. Letzterer wird im Rahmen dieser Thesis genutzt, um Daten von HBASE zu lesen und zu schreiben.\footnote{Siehe \url{https://github.com/apache/hbase/tree/master/hbase-spark}, Stand: 15.6.2018 und deren Nutzung im Projekt \textit{foam-processing-spark} unter \url{https://github.com/jobusam/foam-processing-spark}, Stand: 16.5.2018.}

\subsection{Praxisbeispiele und deren Optimierungen}
Gerade bei der Verarbeitung großer Datenmengen und unter Berücksichtigung des Prinzips der Datenlokalität existieren einige Fallstricke und Hürden bei der Implementierung der Datenverarbeitung. Im Hadoop-Umfeld und bei der Entwicklung im Spark-Context geht es nicht nur um die Art und Weise, wie die Algorithmen auf die Daten angewendet werden, sondern in erster Linie auch immer darum \textbf{wo} die einzelnen Programmteile ausgeführt werden. 
Der Entwickler sollte immer wissen, in welchem Verarbeitungskontext er sich befindet. Zu dieser Problematik werden in diesem Kapitel einige Beispiele herausgegriffen, welche während der Bearbeitung dieser Thesis aufgetreten sind.\\

\noindent
\subsubsection*{Weniger ist mehr TODO}
Ungenutzte Daten so früh wie möglich aus der Verarbeitung rausnehmen. Siehe
Problematik beim HBASE-Spark Connector. Entweder ich mache einen Full-Table Scan und fodere
alle Daten an, um sie später im Spark-Executor auszuführen, oder ich versuche schon beim Zugriff der Daten in den Region-Server mit ColumnFamilies und Filter-Operationen nur die Daten anzufordern, welche auch wirklich benötigt werden. 
\subsubsection*{Caching - Performanz vs. Ressourcen TODO}
Hashing-Problem. Ist es geschickter Daten zu Cachen anstatt sie zweifach anzufordern?
Funktioniert Caching überhaupt mit nicht serializierbaren Daten?
\subsubsection*{Faulheit ist der Schlüssel zum Erfolg TODO}
Lazy-Loading und Ausführung bei RDDs

\subsubsection*{Teile und Herrsche TODO}
Balancing and Repartitionieren. Aufteilung der Last zu gleichen Teilen!
Gerade beim Ausprobieren und Testen ist es einfach, die Resultate eines RDDs nach der Datenverarbeitung über eine Konsole auszugeben. Doch hierbei muss genau überlegt werden, wie diese Resultate ausgegeben werden (siehe Listing \ref{lst:spark_rdd_collect}). In der ersten Variante wird auf dem RDD die Methode collect() aufgerufen und die daraus erhaltene Liste von Objekten wird über ein Logger-Objekt in das Log-File dieser Ausführung geschrieben.\\
Auf der ersten Blick ist aber nicht ersichtlich, was diese Methode wirklich bewirkt. Wie bereits in Kapitel \ref{sec:theory_spark} (TODO: check reference) beschrieben, wird bei der Ausführung einer Spark-Anwendung ein sogenannter Spark-Driver gestartet. Dieser wiederum fordert eine gewisse Anzahl von Exekutoren an, die die eigentlich Datenverarbeitung übernehmen (Master-Slave-Prinzip). Hierbei laufen die Executoren auf einzelnen Knoten innerhalb des Clusters. In dem Moment, in welchem die collect()-Methode auf einem RDD ausgeführt wird, werden die Daten des RDDs \textit{eingesammelt}. Dies bedeutet, dass die Daten des RDDs, welche vorher verteilt auf allen Executoren im Arbeitsspeicher geladen wurden, nun jetzt an den Spark-Driver geschickt werden. Dieser sammelt sozusagen die Ergebnisse der Executoren ein. Diese Mechanismus ist an sich nicht problematisch und funktioniert auch gerade beim Testen mit kleinen Datenmengen. Bei großen RDDs hingegen, werden auch wieder alle Daten an der Driver geschickt und in den meisten Fällen wird dies den begrenzten Arbeitsspeicher des Drivers überfordern. Die Applikation wird mit einer OutOfMemoryException??? beendet!.\\

\noindent
Daher ist es sinnvoll auch schon beim Testen mit kleinen RDDs vorzugsweise die take()-Methode zu nutzen. Diese tut das gleiche wie, die collect()-Methode mit dem Zusatz, dass sie nur ein bestimmte Anzahl von Einträgen sammelt. Dadurch wird selbst bei größeren RDDs der Speicher nicht ausgehen. 


\begin{lstlisting}[label={lst:spark_rdd_collect},caption= Spark Java RDD collect()-Methode ,captionpos=b,frame=single,style=customshell]

HbaseReader hbr = new HbaseReader(jsc, hbaseConfigFile);
JavaRDD<Metadata> forensicMetadata = hbr.getForensicMetadata();

# use collect() method
forensicMetadata.collect().stream()
	.forEach(m -> LOGGER.info("Entry = .", m));

# use take(int amount) method	
forensicMetadata.take(10).stream()
	.forEach(m -> LOGGER.info("Entry = .", m));	
\end{lstlisting}


\section{Anwendungsfälle der Datenverarbeitung}
\subsection{Hashsummen ermitteln}

\subsection{Dateityp erkennen mit Apache Tika}

\subsection{Dateien indizieren}
Ein weitere Anwendungsfall ist die Indizierung von Texten und Wörtern, welche aus den einzelnen Dateien extrahiert wurden.\\

\noindent
Der Grund für eine Indizierung dieser Inhalte ist eine schnellere Suche nach beliebigen Wörtern, als bei der reinen Suche in HBASE. Zur Indizierung existieren zwei bekannte Projekt. Diese sind einerseits \textit{Apache Solr} und andererseits \textit{Elasticsearch}. Bei bauen wiederum auf das \textit{Apache Lucene}-Projekt auf. Apache Solr ist ein Open-Source Projekt unter dem Dach der Apache Foundation. Wohingegen Elasticsearch auch als Open-Source Projekt entwickelt wird, jedoch primär von der kommerziellen Firma \textit{Elastic???} verwaltet wird. Diese bietet gerade Zusatzpakete und Support gegen Bezahlung an.
\subsubsection{Apache Solr}
Spark-Connector von Databracks vorhanden. Der Connector selbst bietet aber vorzugsweise lesenden Zugriff.
\subsubsection{Elasticsearch}
Spark-Connector auf RDD-Basis vorhanden. Kerberos-Sicherung nur über X-Pack gegen Geld vorhanden. Rack-Awareness und Verteilung nicht klar. Elasticsearch baut eigene Infrastruktur auf. Nicht über Ambari steuerbar.
\section{Forensische Anforderungen}

\subsection{Plattform absichern}
Ursprünglich spielt das Thema der Datensicherheit bei Apache Hadoop keine Rolle und gewann erst nach und nach an Relevanz. Anfänglich wurde immer angenommen, dass das Hadoop-Clusters aus vertrauenswürdigen Maschinen besteht, welche von vertrauenswürdigen Nutzern in abgesicherten Umgebungen verwendet wird\footnote{Vgl. \url{https://www.infoq.com/articles/HadoopSecurityModel}.}. Mittlerweile hat sich der Bedarf nach Sicherheit deutlich erhöht, da oftmals rießige vertrauliche Datensätze verarbeitet werden, welche bei Angriffen sehr schnell abfließen könnten.\\

\noindent
Todo:
Das Absichern des Hadoop-Clusters bezieht sich primär auf die Nutzung von Kerberos zur Authentifizierung.Es gibt etliche weitere Projekte, wie beispielsweise Apache Ranger, Apache Atlas und Apache Knox.Sie alle adressieren einen bestimmten Aspekt zur Verbesserung der Systemsicherheit. Allerdings werde ich mich hauptsächlich auf den Einsatz von Kerberos beschränken und prüfen, welche Vorteile diese Lösung bietet und welche Probleme dabei auftauchen können. Darüber hinaus ist es meines Wissens auch möglich, die Daten auf logischer Ebene zu verschlüsseln (im verteilten Dateisystem HDFS). Dies würde einen unbefugten physischen Zugriff erschweren. Diesen Punkt werde ich für die Thesis als optionales Arbeitspaket im Hinterkopf behalten. Wahrscheinlich werde ich mit den anderen Themen aber schon genügend Arbeit haben. 
\subsubsection{Authentifizierung}
Standardmäßig wird Hadoop in Kombination mit Kerberos verwendet, um eine allgemeinen Zugriffsschutz zu ermöglichen.\cite{hadoop_security}\\
Eine Alternative könnte hier auch Cloudera Sentry, Apache Ranger, Apache Atlas oder Apache Knox\footnote{https://knox.apache.org/} sein.
\subsubsection{Datenverschlüsselung}
Prinzipiell lässt sich die Datenverschüsselung in die Szenarien \textit{Persistenzverschlüsselung} und \textit{Transportverschlüsselung} unterteilen. Das HDFS bietet eine Verschlüsselung an, wobei die Komplexität bei Key-Management liegt. Denn schließlich kann ein Hadoop-Cluster mehrere hundert Knoten mit jeweils mehreren Datenträger enthalten. Sie alle müssten eigene Verschlüsselungschlüssel nutzen. Die Verschlüsselung selbst kann direkt auf Betriebsyssystemebene beispielsweise auf LUKS aufbauen, oder sie findet auf logischer Ebene im HDFS statt.\cite{hadoop_security}\\

\noindent
Darüber hinaus ist die Transportverschlüsselung auch möglich. So müssen die einzelnen Services wie Webzugriffe  mit TLS verschlüsselt werden.\footnote{Weiter Infos unter: \url{https://www.infoq.com/articles/HadoopSecurityModel} und\\ \url{https://community.hortonworks.com/articles/102957/hadoop-security-concepts.html}.}\\


\noindent
Letztlich stellt sich die Frage, welche Angriffe den mit Datenverschüsselung vermieden werden sollen. 
\chapter{Datenvisualisierung}
\label{ch:data_visualization}
TODO: 2-3 Seiten über mögliche Visualisierungen schreiben. Hauptsächlich theoretische Aspekte. Kurze Erklärung wie das Projekt \textit{Banana for Solr} genutzt wurde um eine
einfache UI bereitzustellen.\footnote{Siehe Link: \url{https://github.com/lucidworks/banana}. Letzter Zugriff: 23.8.2018.}

\noindent
Vergleich zu Autopsy

Duplicate Flag in DB schreiben, um über Banana UI alle Duplikate finden zu können...

Möglichkeiten und Ideen zur Datenvisualisierung:
\begin{itemize}
\item Für jede Datei sollen Name, Pfad, Größe,Hashsumme, Dateityp, Owner und Group, Zugriffsrechte und die Zeitstempel der Erstellung und letzter Speicherung angezeigt werden. 
\item Nach all diesen Parametern kann auch gesucht werden.
\item Auffinden von Duplikaten anhand der Hashsummen
\item Indizierung für schnelle textbasierte Inhaltssuche?
\item Zeitleiste? (wohl eher optional)
\item Wordcloud, geographische Visualisierung, Flare-Chart, Tree-Map, Calendar-Chart als Timeline?
\item Webframeworks wie \url{https://d3js.org/} \footnote{Siehe Links: \url{https://bl.ocks.org/mbostock/4063550}, \url{https://bl.ocks.org/mbostock/5944371}, \url{https://bl.ocks.org/mbostock/1046712}, \url{https://bl.ocks.org/mbostock/4063269} und \url{http://xliberation.com/googlecharts/d3concept.html}. Letzter Zugriff: 25.7.2018}
%Obige Links wären sehr interessant for die foAm.
\item Neo4j
\item Open Source Community Variante Helical Insight
\item Apache Superset für Visualisierung (siehe Ambari Cluster Services)
\item Apache Grafana?
\item GoJs incremental tree?
\end{itemize}


%--------------------------------------------------------------------------------------
%--------------------------------------------------------------------------------------
%--------------------------------------------------------------------------------------
\chapter{Zusammenfassung}
\label{ch:zusammenfassung}

%\noindent
%Test\footnote{Siehe Link unter \url{https://packea.com/512/.html}. Letzter Zugriff: 29.9.2017.}\\
%
%
%
%\noindent
%Listing \ref{lst:install_commands} zeit tools.
%\begin{lstlisting}[label={lst:install_commands},caption= Befehle zur Einrichtung der Build-Umgebung,captionpos=b,frame=single]
%// Kernel-Updates installieren mithilfe des Paketmanagers dnf
%$ sudo dnf update kernel*
%\end{lstlisting}

%Listing \ref{lst:makefile} zeigt das Makefile zum Bauen des Kernelmoduls. 
%\lstinputlisting[label={lst:makefile},caption= Makefile zum Bauen des Kernelmoduls,captionpos=b,style=customc]{resource/Makefile}

%\lstinputlisting[label={lst:km1_init},caption= Initialisierung und Aufräumen des Moduls,captionpos=b,style=customc,firstline=67,lastline=84]{resource/km_1.c}



\chapter{Ausblick}
\label{ch:ausblick}
Im praktischen Teil

%---------------------------------------------------
%Verzeichnise
\printbibliography%[heading=bibintoc] -> option prints section in content list
\listoffigures
\listoftables
\lstlistoflistings
%---------------------------------------------------
\appendix

\chapter{Allgemeines}

\section{Analyse ähnlicher Projekte und Produkte}
Im Bereich der IT-Sicherheit und Incident Response für Unternehmensinfrastrukturen, existiert das Apache Projekt \textit{Metron}, welches auf dem Hadoop-Ökosystem aufbaut.\footnote{Siehe Link: \url{https://metron.apache.org/}. Letzter Zugriff: 5.3.2018.}\\ Ziel dieses Projektes ist es, Sicherheitsvorfälle zu finden und zu analysieren. Hierbei kann Apache Metron auch mit Telemetriedaten umgehen.\footnote{Siehe Link: \url{https://www.heise.de/developer/meldung/Cybersecurity-Apache-Metron-wird-Top-Level-Projekt-3695901.html}. Letzter Zugriff: 5.3.2018.}\\
Eine entsprechende Abgrenzung zu diesem Projekt besteht aufgrund der unterschiedlichen Projektziele. Diese Thesis bezieht sich auf die forensische Analyse von Asservaten und informationstechnischen Systemen. Es ist nicht das Ziel, Sicherheitsvorfälle in unternehmenskritischen Infrastrukturen zu analysieren.\\

\noindent
Das Open-Source Framework \textit{Turbinia} ist ein weiteres Projekt, welches ähnliche Ziele verfolgt.\footnote{Siehe Link: \url{https://github.com/google/turbinia}. Letzter Zugriff: 5.3.2018.}
Der Grundgedanke ist die Automatisierung und Skalierung forensischer Analysen in Computer-Clustern. Prinzipiell hat dieses Projekt das gleiche Ziel, wie diese Masterthesis. Aufwendige Analysen sollen parallelisiert prozessiert werden, um sie schneller zu verarbeiten. Das Projekt wird derzeit aktiv gepflegt.\footnote{Dies ist daran erkennbar, dass der letzte Commit in das Github-Repository am 4.10.2018 erfolgte (Letzter Zugriff: 6.10.2018).} 
Allerdings ist es noch in einer frühen Alpha-Phase und daher noch nicht ausgereift. Dieses Projekt basiert auch auf einer Master-Client Architektur. Es bietet aber keine Nutzung auf Basis eines verteilten Dateisystems an. Es muss dafür gesorgt werden, dass jeder Knoten auf alle verfügbaren Daten, beziehungsweise Asservate, zugreifen kann.\\  
Im Rahmen dieser Thesis hingegen, wird durch die Nutzung von Apache Hadoop, eine verteilte Speicherung von Daten unterstützt. Darüber hinaus wird die Datenverarbeitung dort ausgeführt, wo die Daten liegen und nicht umgekehrt. \\

\noindent
Ein klassisches Analyse-Werkzeug in der Forensik ist \textit{Autopsy}. Es basiert auf \textit{The Sleuth Kit} und ist kostenlos.\footnote{Siehe Link: \url{https://www.sleuthkit.org/autopsy/}. Letzter Zugriff: 5.3.2018.} Mit dem Werkzeug können Hashsummen berechnet oder auch Multimediadateien analysiert werden. Autopsy ist ein Single-Node Analyseprogramm und läuft vorzugsweise auf einem eigenen Analyserechner pro Nutzer.\\ 
Es gibt auch die Möglichkeit, das Programm kollaborativ zu verwenden. Dabei gibt es einen zentralen Netzwerkspeicher, welcher alle Beweismittel enthält. Es ist möglich, mit mehreren Nutzern parallel am gleichen Fall zu arbeiten und Analyseergebnisse in Echtzeit zu teilen. Diese Art der verteilten Analyse zeigt Ähnlichkeiten zu dieser Thesis auf.\\
Allerdings geht es bei diesem kollaborativen Ansatz vielmehr darum, an einem großen Fall mit mehreren Nutzern zu arbeiten und Ergebnisse einfacher zusammenzutragen. Einzelne Analysen finden nur auf einem konkreten Analyserechner statt. Eine parallele Verarbeitung mittels horizontaler Skalierung wird durch die Anzahl parallel arbeitender Nutzer geschaffen. Jedoch kann das System nicht automatisiert einzelne Analysen auf allen verfügbaren Knoten verarbeiten, wie es in dieser Thesis geplant ist.\\

\noindent
Autopsy bietet keine Möglichkeiten, forensische Analysen im Cluster durchzuführen. Allerdings gibt es eine Variante des \textit{The Sleuth Kit}s, welche das gleiche Ziel verfolgt, wie in dieser Thesis. Hierbei wird die Funktionalität des \textit{The Sleuth Kit}s in einen Apache Hadoop Cluster übertragen.\footnote{Siehe Link: \url{https://www.sleuthkit.org/tsk_hadoop/index.php}. Letzter Zugriff: 6.10.2018.} Das Projekt nutzt Apache Hadoop und auch Apache HBase zur Speicherung von Datenträgern im Cluster. Zur Prozessierung der Daten wird allerdings nicht Apache Spark genutzt, sondern das Hadoop interne Map-Reduce Verfahren. Darüber hinaus wird das Projekt seit 2012 nicht mehr gepflegt.\footnote{Siehe Link zum Source-Code Repository auf GitHub unter: \url{https://github.com/sleuthkit/hadoop_framework}. Letzter Zugriff: 6.10.2018.} Es ist nicht bekannt, aus welchen Gründen die Datenverarbeitung im Cluster eingestellt wurde.

\section{Lizenzierungen in dieser Arbeit}
\label{sec:licencing_issues}
Die dargestellten Gantt-Diagramme (siehe Abbildungen \ref{fig:ganttA}, \ref{fig:ganttB}, \ref{fig:ganttC}) wurden mit der JavaScript-Bibliothek \textit{dhtmlxGantt} erstellt.\footnote{Siehe Link: \url{https://github.com/DHTMLX/gantt}. Letzter Zugriff: 20.04.2018.} Der Quellcode ist unter der \textit{GNU GPLv2}-Lizenz lizenziert. Die aktuelle Bibliothek kann unter folgendem Link: \url{https://dhtmlx.com/docs/products/dhtmlxGantt/download.shtml} heruntergeladen werden (Letzter Zugriff: 21.3.2018).\\


\noindent
Nachfolgend werden die Logos aufgelistet, welche in Abbildung \ref{fig:hadoop_framework_structure} dargestellt werden. Die Logos der Projekte und die Projektnamen sind Handelsmarken der Apache Source Foundation.\footnote{Siehe Link: \url{https://www.apache.org/}. Letzter Zugriff: 20.5.2018.} Sie dürfen in Publikationen genutzt werden.\footnote{Siehe Link: \url{https://www.apache.org/foundation/marks/}. Letzter Zugriff: 21.3.2018.}

\begin{itemize}
\item Apache Ambari\texttrademark\thinspace Logo von \url{https://ambari.apache.org/}. Letzter Zugriff: 21.3.2018. 
\item Apache Hadoop\textsuperscript{\textregistered} Logo von \url{https://hadoop.apache.org/}. Letzter Zugriff: 21.3.2018.
\item Apache Spark\texttrademark\thinspace Logo von \url{https://spark.apache.org/}. Letzter Zugriff: 21.3.2018.
\item Apache HBase\textsuperscript{\textregistered} Logo von \url{https://hbase.apache.org/}. Letzter Zugriff: 21.3.2018.
\item Apache Hive\texttrademark\thinspace Logo von \url{https://hive.apache.org/}. Letzter Zugriff: 21.3.2018.
\item Apache ZooKeeper\texttrademark\thinspace Logo von \url{https://zookeeper.apache.org/}. Letzter Zugriff: 21.3.2018.
\end{itemize}


\chapter{Datenimport}
\section{Konfigurationsdateien}
\label{sec:appendix_data_import_config_management}
Nachfolgend werden die minimalen Konfigurationsdateien dargestellt, welche beim Datenimport angegeben werden müssen. Diese werden genutzt, um die Verbindung zum HDFS-Dateisystem und zur HBase-Datenbank zu definieren. Letztlich muss in beiden Konfigurationen der Service-Endpunkt konfiguriert werden.
\lstinputlisting[label={lst:hdfs_core_xml},caption= Minimale Konfiguration der Datei \textit{hdfs-core.xml},captionpos=b,style=customshell]{resource/appendix/hdfs-core.xml}

\lstinputlisting[label={lst:hbase_site_xml},caption= Minimale Konfiguration der Datei \textit{hbase-site.xml},captionpos=b,style=customshell]{resource/appendix/hbase-site.xml}


%\chapter{Hadoop Konfigurationen}
%\section{Aufsetzen des aktuellen Hadoop-Frameworks}
%Listing \ref{lst:config_hadoop} zeigt die Schritte zum Konfigurieren des Hadoop-Frameworks
%\lstinputlisting[label={lst:config_hadoop},caption= Konfiguration des Hadoop-Frameworks,captionpos=b,style=customshell]{resource/appendix/hadoop_configuration.txt}

\end{document}
