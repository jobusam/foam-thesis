\chapter{Visualisierung der Ergebnisse}
Ziel dieser forensischen Analyseplattform ist es, dem Nutzer einen Überblick bei der Datensichtung zu geben. Hierbei ist es essentiell entsprechende Visualisierungen zu verwenden.
Welche Ziele sollen erreicht werden?
\begin{itemize}
\item Für jede Datei sollen Name, Pfad, Größe,Hashsumme, Dateityp, Owner und Group, Zugriffsrechte und die Zeitstempel der Erstellung und letzter Speicherung angezeigt werden. 
\item Nach all diesen Parametern kann auch gesucht werden.
\item Auffinden von Duplikaten anhand der Hashsummen
\item Indizierung für schnelle textbasierte Inhaltssuche?
\item Zeitleiste? (wohl eher optional)
\item Wordcloud, geographische Visualisierung, Flare-Chart, Tree-Map, Calendar-Chart als Timeline?
\item Webframeworks wie \url{https://d3js.org/} \footnote{Siehe auch \url{https://bl.ocks.org/mbostock/4063550} oder \url{https://bl.ocks.org/mbostock/5944371} oder \url{https://bl.ocks.org/mbostock/1046712} oder \url{https://bl.ocks.org/mbostock/4063269}. Letzteres wäre characteristisch für foAm. oder \url{http://xliberation.com/googlecharts/d3concept.html}}
\item Neo4j
\item Open Source Community Variante Helical Insight
\item Apache Superset für Visualisierung (siehe Ambari Cluster Services)
\item Apache Grafana?
\item GoJs incremental tree?
\end{itemize}
