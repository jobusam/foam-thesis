\chapter{Datenvisualisierung}
\label{ch:data_visualization}
TODO: 2-3 Seiten über mögliche Visualisierungen schreiben. Hauptsächlich theoretische Aspekte. Kurze Erklärung wie das Projekt \textit{Banana for Solr} genutzt wurde um eine
einfache UI bereitzustellen.\footnote{Siehe Link: \url{https://github.com/lucidworks/banana}. Letzter Zugriff: 23.8.2018.}

\noindent
Vergleich zu Autopsy

Duplicate Flag in DB schreiben, um über Banana UI alle Duplikate finden zu können...

Möglichkeiten und Ideen zur Datenvisualisierung:
\begin{itemize}
\item Für jede Datei sollen Name, Pfad, Größe,Hashsumme, Dateityp, Owner und Group, Zugriffsrechte und die Zeitstempel der Erstellung und letzter Speicherung angezeigt werden. 
\item Nach all diesen Parametern kann auch gesucht werden.
\item Auffinden von Duplikaten anhand der Hashsummen
\item Indizierung für schnelle textbasierte Inhaltssuche?
\item Zeitleiste? (wohl eher optional)
\item Wordcloud, geographische Visualisierung, Flare-Chart, Tree-Map, Calendar-Chart als Timeline?
\item Webframeworks wie \url{https://d3js.org/} \footnote{Siehe Links: \url{https://bl.ocks.org/mbostock/4063550}, \url{https://bl.ocks.org/mbostock/5944371}, \url{https://bl.ocks.org/mbostock/1046712}, \url{https://bl.ocks.org/mbostock/4063269} und \url{http://xliberation.com/googlecharts/d3concept.html}. Letzter Zugriff: 25.7.2018}
%Obige Links wären sehr interessant for die foAm.
\item Neo4j
\item Open Source Community Variante Helical Insight
\item Apache Superset für Visualisierung (siehe Ambari Cluster Services)
\item Apache Grafana?
\item GoJs incremental tree?
\end{itemize}
